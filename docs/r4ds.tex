% Options for packages loaded elsewhere
\PassOptionsToPackage{unicode}{hyperref}
\PassOptionsToPackage{hyphens}{url}
%
\documentclass[
]{latex/krantz}
\usepackage{amsmath,amssymb}
\usepackage{iftex}
\ifPDFTeX
  \usepackage[T1]{fontenc}
  \usepackage[utf8]{inputenc}
  \usepackage{textcomp} % provide euro and other symbols
\else % if luatex or xetex
  \usepackage{unicode-math} % this also loads fontspec
  \defaultfontfeatures{Scale=MatchLowercase}
  \defaultfontfeatures[\rmfamily]{Ligatures=TeX,Scale=1}
\fi
\usepackage{lmodern}
\ifPDFTeX\else
  % xetex/luatex font selection
\fi
% Use upquote if available, for straight quotes in verbatim environments
\IfFileExists{upquote.sty}{\usepackage{upquote}}{}
\IfFileExists{microtype.sty}{% use microtype if available
  \usepackage[]{microtype}
  \UseMicrotypeSet[protrusion]{basicmath} % disable protrusion for tt fonts
}{}
\makeatletter
\@ifundefined{KOMAClassName}{% if non-KOMA class
  \IfFileExists{parskip.sty}{%
    \usepackage{parskip}
  }{% else
    \setlength{\parindent}{0pt}
    \setlength{\parskip}{6pt plus 2pt minus 1pt}}
}{% if KOMA class
  \KOMAoptions{parskip=half}}
\makeatother
\usepackage{xcolor}
\usepackage{color}
\usepackage{fancyvrb}
\newcommand{\VerbBar}{|}
\newcommand{\VERB}{\Verb[commandchars=\\\{\}]}
\DefineVerbatimEnvironment{Highlighting}{Verbatim}{commandchars=\\\{\}}
% Add ',fontsize=\small' for more characters per line
\usepackage{framed}
\definecolor{shadecolor}{RGB}{248,248,248}
\newenvironment{Shaded}{\begin{snugshade}}{\end{snugshade}}
\newcommand{\AlertTok}[1]{\textcolor[rgb]{0.94,0.16,0.16}{#1}}
\newcommand{\AnnotationTok}[1]{\textcolor[rgb]{0.56,0.35,0.01}{\textbf{\textit{#1}}}}
\newcommand{\AttributeTok}[1]{\textcolor[rgb]{0.13,0.29,0.53}{#1}}
\newcommand{\BaseNTok}[1]{\textcolor[rgb]{0.00,0.00,0.81}{#1}}
\newcommand{\BuiltInTok}[1]{#1}
\newcommand{\CharTok}[1]{\textcolor[rgb]{0.31,0.60,0.02}{#1}}
\newcommand{\CommentTok}[1]{\textcolor[rgb]{0.56,0.35,0.01}{\textit{#1}}}
\newcommand{\CommentVarTok}[1]{\textcolor[rgb]{0.56,0.35,0.01}{\textbf{\textit{#1}}}}
\newcommand{\ConstantTok}[1]{\textcolor[rgb]{0.56,0.35,0.01}{#1}}
\newcommand{\ControlFlowTok}[1]{\textcolor[rgb]{0.13,0.29,0.53}{\textbf{#1}}}
\newcommand{\DataTypeTok}[1]{\textcolor[rgb]{0.13,0.29,0.53}{#1}}
\newcommand{\DecValTok}[1]{\textcolor[rgb]{0.00,0.00,0.81}{#1}}
\newcommand{\DocumentationTok}[1]{\textcolor[rgb]{0.56,0.35,0.01}{\textbf{\textit{#1}}}}
\newcommand{\ErrorTok}[1]{\textcolor[rgb]{0.64,0.00,0.00}{\textbf{#1}}}
\newcommand{\ExtensionTok}[1]{#1}
\newcommand{\FloatTok}[1]{\textcolor[rgb]{0.00,0.00,0.81}{#1}}
\newcommand{\FunctionTok}[1]{\textcolor[rgb]{0.13,0.29,0.53}{\textbf{#1}}}
\newcommand{\ImportTok}[1]{#1}
\newcommand{\InformationTok}[1]{\textcolor[rgb]{0.56,0.35,0.01}{\textbf{\textit{#1}}}}
\newcommand{\KeywordTok}[1]{\textcolor[rgb]{0.13,0.29,0.53}{\textbf{#1}}}
\newcommand{\NormalTok}[1]{#1}
\newcommand{\OperatorTok}[1]{\textcolor[rgb]{0.81,0.36,0.00}{\textbf{#1}}}
\newcommand{\OtherTok}[1]{\textcolor[rgb]{0.56,0.35,0.01}{#1}}
\newcommand{\PreprocessorTok}[1]{\textcolor[rgb]{0.56,0.35,0.01}{\textit{#1}}}
\newcommand{\RegionMarkerTok}[1]{#1}
\newcommand{\SpecialCharTok}[1]{\textcolor[rgb]{0.81,0.36,0.00}{\textbf{#1}}}
\newcommand{\SpecialStringTok}[1]{\textcolor[rgb]{0.31,0.60,0.02}{#1}}
\newcommand{\StringTok}[1]{\textcolor[rgb]{0.31,0.60,0.02}{#1}}
\newcommand{\VariableTok}[1]{\textcolor[rgb]{0.00,0.00,0.00}{#1}}
\newcommand{\VerbatimStringTok}[1]{\textcolor[rgb]{0.31,0.60,0.02}{#1}}
\newcommand{\WarningTok}[1]{\textcolor[rgb]{0.56,0.35,0.01}{\textbf{\textit{#1}}}}
\usepackage{longtable,booktabs,array}
\usepackage{calc} % for calculating minipage widths
% Correct order of tables after \paragraph or \subparagraph
\usepackage{etoolbox}
\makeatletter
\patchcmd\longtable{\par}{\if@noskipsec\mbox{}\fi\par}{}{}
\makeatother
% Allow footnotes in longtable head/foot
\IfFileExists{footnotehyper.sty}{\usepackage{footnotehyper}}{\usepackage{footnote}}
\makesavenoteenv{longtable}
\usepackage{graphicx}
\makeatletter
\def\maxwidth{\ifdim\Gin@nat@width>\linewidth\linewidth\else\Gin@nat@width\fi}
\def\maxheight{\ifdim\Gin@nat@height>\textheight\textheight\else\Gin@nat@height\fi}
\makeatother
% Scale images if necessary, so that they will not overflow the page
% margins by default, and it is still possible to overwrite the defaults
% using explicit options in \includegraphics[width, height, ...]{}
\setkeys{Gin}{width=\maxwidth,height=\maxheight,keepaspectratio}
% Set default figure placement to htbp
\makeatletter
\def\fps@figure{htbp}
\makeatother
\setlength{\emergencystretch}{3em} % prevent overfull lines
\providecommand{\tightlist}{%
  \setlength{\itemsep}{0pt}\setlength{\parskip}{0pt}}
\setcounter{secnumdepth}{5}
\usepackage[portuguese]{babel}

\usepackage{booktabs}
\usepackage{longtable}
\usepackage[bf,singlelinecheck=off]{caption}

\usepackage{Alegreya}
\usepackage[scale=.7]{sourcecodepro}

\usepackage{framed,color}
\definecolor{shadecolor}{RGB}{248,248,248}

\renewcommand{\textfraction}{0.05}
\renewcommand{\topfraction}{0.8}
\renewcommand{\bottomfraction}{0.8}
\renewcommand{\floatpagefraction}{0.75}

\renewenvironment{quote}{\begin{VF}}{\end{VF}}
\usepackage{hyperref}
\let\oldhref\href
\renewcommand{\href}[2]{#2\footnote{\url{#1}}}

\ifxetex
  \usepackage{letltxmacro}
  \setlength{\XeTeXLinkMargin}{1pt}
  \LetLtxMacro\SavedIncludeGraphics\includegraphics
  \def\includegraphics#1#{% #1 catches optional stuff (star/opt. arg.)
    \IncludeGraphicsAux{#1}%
  }%
  \newcommand*{\IncludeGraphicsAux}[2]{%
    \XeTeXLinkBox{%
      \SavedIncludeGraphics#1{#2}%
    }%
  }%
\fi

\makeatletter
\newenvironment{kframe}{%
\medskip{}
\setlength{\fboxsep}{.8em}
 \def\at@end@of@kframe{}%
 \ifinner\ifhmode%
  \def\at@end@of@kframe{\end{minipage}}%
  \begin{minipage}{\columnwidth}%
 \fi\fi%
 \def\FrameCommand##1{\hskip\@totalleftmargin \hskip-\fboxsep
 \colorbox{shadecolor}{##1}\hskip-\fboxsep
     % There is no \\@totalrightmargin, so:
     \hskip-\linewidth \hskip-\@totalleftmargin \hskip\columnwidth}%
 \MakeFramed {\advance\hsize-\width
   \@totalleftmargin\z@ \linewidth\hsize
   \@setminipage}}%
 {\par\unskip\endMakeFramed%
 \at@end@of@kframe}
\makeatother

\makeatletter
\@ifundefined{Shaded}{
}{\renewenvironment{Shaded}{\begin{kframe}}{\end{kframe}}}
\makeatother

\newenvironment{rmdblock}[1]
  {
  \begin{itemize}
  \renewcommand{\labelitemi}{
    \raisebox{-.7\height}[0pt][0pt]{
      {\setkeys{Gin}{width=3em,keepaspectratio}\includegraphics{images/#1}}
    }
  }
  \setlength{\fboxsep}{1em}
  \begin{kframe}
  \item
  }
  {
  \end{kframe}
  \end{itemize}
  }
\newenvironment{rmdnote}
  {\begin{rmdblock}{note}}
  {\end{rmdblock}}
\newenvironment{rmdcaution}
  {\begin{rmdblock}{caution}}
  {\end{rmdblock}}
\newenvironment{rmdimportant}
  {\begin{rmdblock}{important}}
  {\end{rmdblock}}
\newenvironment{rmdtip}
  {\begin{rmdblock}{tip}}
  {\end{rmdblock}}
\newenvironment{rmdwarning}
  {\begin{rmdblock}{warning}}
  {\end{rmdblock}}

\usepackage{makeidx}
\makeindex

\urlstyle{tt}

\usepackage{amsthm}
\makeatletter
\def\thm@space@setup{%
  \thm@preskip=8pt plus 2pt minus 4pt
  \thm@postskip=\thm@preskip
}
\makeatother

\frontmatter
\ifLuaTeX
  \usepackage{selnolig}  % disable illegal ligatures
\fi
\usepackage[]{natbib}
\bibliographystyle{plainnat}
\IfFileExists{bookmark.sty}{\usepackage{bookmark}}{\usepackage{hyperref}}
\IfFileExists{xurl.sty}{\usepackage{xurl}}{} % add URL line breaks if available
\urlstyle{same}
\hypersetup{
  pdftitle={R para Data Science},
  pdfauthor={Jeidsan A. da C. Pereira},
  hidelinks,
  pdfcreator={LaTeX via pandoc}}

\title{R para Data Science}
\usepackage{etoolbox}
\makeatletter
\providecommand{\subtitle}[1]{% add subtitle to \maketitle
  \apptocmd{\@title}{\par {\large #1 \par}}{}{}
}
\makeatother
\subtitle{Solução dos exercícios}
\author{Jeidsan A. da C. Pereira}
\date{2023-10-25}

\usepackage{amsthm}
\newtheorem{theorem}{Teorema}[chapter]
\newtheorem{lemma}{Lema}[chapter]
\newtheorem{corollary}{Corolário}[chapter]
\newtheorem{proposition}{Proposição}[chapter]
\newtheorem{conjecture}{Conjectura}[chapter]
\theoremstyle{definition}
\newtheorem{definition}{Definição}[chapter]
\theoremstyle{definition}
\newtheorem{example}{Exemplo}[chapter]
\theoremstyle{definition}
\newtheorem{exercise}{Exercício}[chapter]
\theoremstyle{definition}
\newtheorem{hypothesis}{Hipótese}[chapter]
\theoremstyle{remark}
\newtheorem*{remark}{Observação}
\newtheorem*{solution}{Solução}
\begin{document}
\maketitle

%\cleardoublepage\newpage\thispagestyle{empty}\null
%\cleardoublepage\newpage\thispagestyle{empty}\null
%\cleardoublepage\newpage
\thispagestyle{empty}
\begin{center}
\includegraphics{images/dedication.pdf}
\end{center}

\setlength{\abovedisplayskip}{-5pt}
\setlength{\abovedisplayshortskip}{-5pt}

{
\setcounter{tocdepth}{1}
\tableofcontents
}
\hypertarget{prefuxe1cio}{%
\chapter*{Prefácio}\label{prefuxe1cio}}
\addcontentsline{toc}{chapter}{Prefácio}

Esta página serviu para estudo e prática com o pacote R Bookdown e contém a solução encontrada por mim para os exercícios propostos no livro R para Data Sciente, de Hadley Wickham e Garret Grolemund, publicado no Brasil em 2019 pela Alta Books Editora \citep{wickham2019}.

Por se tratar de um produto construído durante o processo de aprendizagem, o conteúdo pode conter erros, tanto no texto em si, como na lógica utilizada para solução dos exercícios.

Dúvidas ou sugestões de melhoria podem ser encaminhadas para o e-mail \emph{\href{mailto:jeidsan.pereira@gmail.com}{\nolinkurl{jeidsan.pereira@gmail.com}}}.

\hypertarget{penduxeancias}{%
\section*{Pendências}\label{penduxeancias}}
\addcontentsline{toc}{section}{Pendências}

\begin{itemize}
\tightlist
\item
  No PDF, o prefácio está sendo exibido duas vezes no sumário;
\item
  \protect\hyperlink{exr1-7-4}{Exercício 1.7.4};
\item
  \protect\hyperlink{exr2-3-3}{Exercício 2.3.3};
\item
\end{itemize}

\mainmatter

\hypertarget{part-explorar}{%
\part{Explorar}\label{part-explorar}}

\hypertarget{visualizauxe7uxe3o-de-dados-com-ggplot2}{%
\chapter{\texorpdfstring{Visualização de dados com \texttt{ggplot2}}{Visualização de dados com ggplot2}}\label{visualizauxe7uxe3o-de-dados-com-ggplot2}}

Para a correta execução dos códigos desse capítulo, utilizaremos algumas configurações específicas.

Inicialmente, precisaremos carregar o pacote \texttt{nycflights13}, que contém os dados de todos os voos da cidade de Nova York em 2013.

\begin{Shaded}
\begin{Highlighting}[]
\FunctionTok{library}\NormalTok{(nycflights13)}
\FunctionTok{library}\NormalTok{(gridExtra)}
\end{Highlighting}
\end{Shaded}

\begin{verbatim}
## 
## Attaching package: 'gridExtra'
\end{verbatim}

\begin{verbatim}
## The following object is masked from 'package:dplyr':
## 
##     combine
\end{verbatim}

\hypertarget{introduuxe7uxe3o}{%
\section{Introdução}\label{introduuxe7uxe3o}}

Não temos exercícios nesta seção.

\hypertarget{primeiros-passos}{%
\section{Primeiros passos}\label{primeiros-passos}}

\hypertarget{exr1-2-1}{%
\subsection*{Exercício 1.2.1}\label{exr1-2-1}}
\addcontentsline{toc}{subsection}{Exercício 1.2.1}

Execute \texttt{ggplot(data=mpg);}. O que você vê?

\begin{solution}
\leavevmode

\begin{Shaded}
\begin{Highlighting}[]
\FunctionTok{ggplot}\NormalTok{(}\AttributeTok{data=}\NormalTok{mpg) }\SpecialCharTok{+}
\NormalTok{    tema}
\end{Highlighting}
\end{Shaded}

\includegraphics{r4ds_files/figure-latex/unnamed-chunk-3-1.pdf}

É exibido um quadro em branco. Este quadro contém o sistema de coordenadas sobre o qual serão desenhados os grpaficos que pretendemos exibir.

\end{solution}

\hypertarget{exr1-2-2}{%
\subsection*{Exercício 1.2.2}\label{exr1-2-2}}
\addcontentsline{toc}{subsection}{Exercício 1.2.2}

Quantas linhas existem em \texttt{mtcars}? Quantas colunas?

\begin{solution}
\leavevmode

\begin{Shaded}
\begin{Highlighting}[]
\FunctionTok{dim}\NormalTok{(mtcars)}
\end{Highlighting}
\end{Shaded}

\begin{verbatim}
## [1] 32 11
\end{verbatim}

R.: Existem 32 linhas e 11 colunas.

\end{solution}

\hypertarget{exr1-2-3}{%
\subsection*{Exercício 1.2.3}\label{exr1-2-3}}
\addcontentsline{toc}{subsection}{Exercício 1.2.3}

O que a variável \texttt{drv} descreve?

\begin{solution}
Executamos o comando \texttt{?mpg} no console no R e a página de ajuda foi aberta. Nela encontramos o significado de cada variável do conjunto de dados.

A varíável descreve o tipo de tração dos carros analisados, onde \texttt{f} significa tração dianteira, \texttt{r} significa tração traseira e \texttt{4} significa tração nas quatro rodas.
\end{solution}

\hypertarget{ex1-2-4}{%
\subsection*{Exercício 1.2.4}\label{ex1-2-4}}
\addcontentsline{toc}{subsection}{Exercício 1.2.4}

Faça um gráfico de dispersão de \texttt{hwy} \emph{versus} \texttt{cyl}.

\begin{solution}
\leavevmode

\begin{Shaded}
\begin{Highlighting}[]
\FunctionTok{ggplot}\NormalTok{(}\AttributeTok{data =}\NormalTok{ mpg) }\SpecialCharTok{+}
    \FunctionTok{geom\_point}\NormalTok{(}\AttributeTok{mapping =} \FunctionTok{aes}\NormalTok{(}\AttributeTok{x =}\NormalTok{ hwy, }\AttributeTok{y =}\NormalTok{ cyl)) }\SpecialCharTok{+}
\NormalTok{    tema}
\end{Highlighting}
\end{Shaded}

\includegraphics{r4ds_files/figure-latex/unnamed-chunk-5-1.pdf}

\end{solution}

\hypertarget{exr1-2-5}{%
\subsection*{Exercício 1.2.5}\label{exr1-2-5}}
\addcontentsline{toc}{subsection}{Exercício 1.2.5}

O que acontece se você fizer um gráfico de dispersão de \texttt{class} \emph{versus} \texttt{drv}? Por que esse gráfico não é útil?

\begin{solution}
\leavevmode

\begin{Shaded}
\begin{Highlighting}[]
\FunctionTok{ggplot}\NormalTok{(}\AttributeTok{data =}\NormalTok{ mpg) }\SpecialCharTok{+}
    \FunctionTok{geom\_point}\NormalTok{(}\AttributeTok{mapping =} \FunctionTok{aes}\NormalTok{(}\AttributeTok{x =}\NormalTok{ drv, }\AttributeTok{y =}\NormalTok{ class)) }\SpecialCharTok{+}
\NormalTok{    tema}
\end{Highlighting}
\end{Shaded}

\includegraphics{r4ds_files/figure-latex/unnamed-chunk-6-1.pdf}

Apesar de serem exibidos dados no gráfico, nenhuma informação substancial é extraída, uma vez que o tipo de tração não está (a princípio) relacionado com a categoria do carro. Outro fator que torno o gráfico pouco informativo é que há, por exemplo, diversas SUVs com tração nas 4 rodas, contudo os valores ficam sobrepostos no gráfico, não dando dimensão do quanto de dados temos.

Abaixo seguem duas opções de como trazer mais informação ao gráfico:

\begin{itemize}
\tightlist
\item
  a primeira opção adiciona um ruído aos dados (\texttt{position\ =\ jitter} ou \texttt{geom\_jitter()}) de modo que não haja sobreposição;
\end{itemize}

\begin{Shaded}
\begin{Highlighting}[]
\FunctionTok{ggplot}\NormalTok{(}\AttributeTok{data =}\NormalTok{ mpg) }\SpecialCharTok{+}
    \FunctionTok{geom\_point}\NormalTok{(}\AttributeTok{mapping =} \FunctionTok{aes}\NormalTok{(}\AttributeTok{x =}\NormalTok{ drv, }\AttributeTok{y =}\NormalTok{ class), }\AttributeTok{position =} \StringTok{"jitter"}\NormalTok{) }\SpecialCharTok{+}
\NormalTok{    tema}
\end{Highlighting}
\end{Shaded}

\includegraphics{r4ds_files/figure-latex/unnamed-chunk-7-1.pdf}

\begin{itemize}
\tightlist
\item
  a segunda opção, bem mais avançada, adiciona uma estética de \texttt{size} considerando a quantidade de registros.
\end{itemize}

\begin{Shaded}
\begin{Highlighting}[]
\NormalTok{mpg }\SpecialCharTok{\%\textgreater{}\%}
    \FunctionTok{group\_by}\NormalTok{(class, drv) }\SpecialCharTok{\%\textgreater{}\%}
    \FunctionTok{summarize}\NormalTok{(}\AttributeTok{count =} \FunctionTok{n}\NormalTok{()) }\SpecialCharTok{\%\textgreater{}\%}
    \FunctionTok{ggplot}\NormalTok{(}\AttributeTok{mapping =} \FunctionTok{aes}\NormalTok{(}\AttributeTok{x =}\NormalTok{ drv, }\AttributeTok{y =}\NormalTok{ class, }\AttributeTok{size =}\NormalTok{ count)) }\SpecialCharTok{+}
        \FunctionTok{geom\_point}\NormalTok{() }\SpecialCharTok{+}
\NormalTok{        tema}
\end{Highlighting}
\end{Shaded}

\begin{verbatim}
## `summarise()` has grouped output by 'class'. You can override using the
## `.groups` argument.
\end{verbatim}

\includegraphics{r4ds_files/figure-latex/unnamed-chunk-8-1.pdf}

\end{solution}

\hypertarget{mapeamentos-estuxe9ticos}{%
\section{Mapeamentos estéticos}\label{mapeamentos-estuxe9ticos}}

\hypertarget{exr1-3-1}{%
\subsection*{Exercício 1.3.1}\label{exr1-3-1}}
\addcontentsline{toc}{subsection}{Exercício 1.3.1}

O que há de errado com este código? Por que os pontos não estão azuis?

\begin{Shaded}
\begin{Highlighting}[]
\FunctionTok{ggplot}\NormalTok{(}\AttributeTok{data =}\NormalTok{ mpg) }\SpecialCharTok{+}
    \FunctionTok{geom\_point}\NormalTok{(}\AttributeTok{mapping =} \FunctionTok{aes}\NormalTok{(}\AttributeTok{x =}\NormalTok{ displ, }\AttributeTok{y =}\NormalTok{ hwy, }\AttributeTok{color =} \StringTok{"blue"}\NormalTok{)) }\SpecialCharTok{+}
\NormalTok{    tema}
\end{Highlighting}
\end{Shaded}

\includegraphics{r4ds_files/figure-latex/unnamed-chunk-9-1.pdf}

\begin{solution}
Ao invés de atribuir uma cor aos elementos de \texttt{geom\_point}, o atributo \texttt{color} foi passado como uma estética. O gráfico deveria ser construído da seguinte maneira:

\begin{Shaded}
\begin{Highlighting}[]
\FunctionTok{ggplot}\NormalTok{(}\AttributeTok{data =}\NormalTok{ mpg) }\SpecialCharTok{+}
    \FunctionTok{geom\_point}\NormalTok{(}\AttributeTok{mapping =} \FunctionTok{aes}\NormalTok{(}\AttributeTok{x =}\NormalTok{ displ, }\AttributeTok{y =}\NormalTok{ hwy), }\AttributeTok{color =} \StringTok{"blue"}\NormalTok{) }\SpecialCharTok{+}
\NormalTok{    tema}
\end{Highlighting}
\end{Shaded}

\includegraphics{r4ds_files/figure-latex/unnamed-chunk-10-1.pdf}
\end{solution}

\hypertarget{exr1-3-2}{%
\subsection*{Exercício 1.3.2}\label{exr1-3-2}}
\addcontentsline{toc}{subsection}{Exercício 1.3.2}

Quais variáveis em \texttt{mpg} são categóricas? Quais variáveis são contínuas? Como você pode ver essa informação quando executa \texttt{mpg}?

\begin{solution}
Usando \texttt{?mpg} vemos que as variáveis categóricas são: \texttt{manufacturer}, \texttt{model}, \texttt{trans}, \texttt{drv}, \texttt{fl} e \texttt{class}. As variáveis contínuas são: \texttt{displ}, \texttt{cty}, \texttt{hwy}.
\end{solution}

\hypertarget{exr1-3-3}{%
\subsection*{Exercício 1.3.3}\label{exr1-3-3}}
\addcontentsline{toc}{subsection}{Exercício 1.3.3}

Mapeie uma variável contínua para \texttt{color}, \texttt{size} e \texttt{shape}. Como essas estéticas se comportam de maneira diferente para variáveis categóricas e contínuas?

\begin{solution}
\leavevmode

\begin{Shaded}
\begin{Highlighting}[]
\FunctionTok{ggplot}\NormalTok{(}\AttributeTok{data =}\NormalTok{ mpg) }\SpecialCharTok{+}
    \FunctionTok{geom\_point}\NormalTok{(}\AttributeTok{mapping =} \FunctionTok{aes}\NormalTok{(}\AttributeTok{x =}\NormalTok{ displ, }\AttributeTok{y =}\NormalTok{ hwy, }\AttributeTok{color =}\NormalTok{ displ)) }\SpecialCharTok{+}
\NormalTok{    tema}
\end{Highlighting}
\end{Shaded}

\includegraphics{r4ds_files/figure-latex/unnamed-chunk-11-1.pdf}

\begin{Shaded}
\begin{Highlighting}[]
\FunctionTok{ggplot}\NormalTok{(}\AttributeTok{data =}\NormalTok{ mpg) }\SpecialCharTok{+}
    \FunctionTok{geom\_point}\NormalTok{(}\AttributeTok{mapping =} \FunctionTok{aes}\NormalTok{(}\AttributeTok{x =}\NormalTok{ displ, }\AttributeTok{y =}\NormalTok{ hwy, }\AttributeTok{size =}\NormalTok{ displ)) }\SpecialCharTok{+}
\NormalTok{    tema}
\end{Highlighting}
\end{Shaded}

\includegraphics{r4ds_files/figure-latex/unnamed-chunk-12-1.pdf}

\begin{Shaded}
\begin{Highlighting}[]
\FunctionTok{ggplot}\NormalTok{(}\AttributeTok{data =}\NormalTok{ mpg) }\SpecialCharTok{+}
    \FunctionTok{geom\_point}\NormalTok{(}\AttributeTok{mapping =} \FunctionTok{aes}\NormalTok{(}\AttributeTok{x =}\NormalTok{ displ, }\AttributeTok{y =}\NormalTok{ hwy, }\AttributeTok{shape =}\NormalTok{ displ)) }\SpecialCharTok{+}
\NormalTok{    tema}
\end{Highlighting}
\end{Shaded}

\begin{verbatim}
## Error in `geom_point()`:
## ! Problem while computing aesthetics.
## i Error occurred in the 1st layer.
## Caused by error in `scale_f()`:
## ! A continuous variable cannot be mapped to the shape aesthetic
## i choose a different aesthetic or use `scale_shape_binned()`
\end{verbatim}

Quando possível, a biblioteca \emph{ggplot} apesenta a estética em um gradiente, como em color e size. Porém, nem sempre isso é possível, como vemos em \texttt{shape}, que só pode ser utilizada com variáveis discretas ou categóricas.

\end{solution}

\hypertarget{exr1-3-4}{%
\subsection*{Exercício 1.3.4}\label{exr1-3-4}}
\addcontentsline{toc}{subsection}{Exercício 1.3.4}

O que acontece se você mapear a mesma variável a várias estéticas?

\begin{solution}
\leavevmode

\begin{Shaded}
\begin{Highlighting}[]
\FunctionTok{ggplot}\NormalTok{(}\AttributeTok{data =}\NormalTok{ mpg) }\SpecialCharTok{+}
    \FunctionTok{geom\_point}\NormalTok{(}\AttributeTok{mapping =} \FunctionTok{aes}\NormalTok{(}\AttributeTok{x =}\NormalTok{ displ, }\AttributeTok{y =}\NormalTok{ hwy, }\AttributeTok{size =}\NormalTok{ class, }\AttributeTok{color =}\NormalTok{ class, }\AttributeTok{shape =}\NormalTok{ class)) }\SpecialCharTok{+}
\NormalTok{    tema}
\end{Highlighting}
\end{Shaded}

\begin{verbatim}
## Warning: Using size for a discrete variable is not advised.
\end{verbatim}

\begin{verbatim}
## Warning: The shape palette can deal with a maximum of 6 discrete values because
## more than 6 becomes difficult to discriminate; you have 7. Consider
## specifying shapes manually if you must have them.
\end{verbatim}

\begin{verbatim}
## Warning: Removed 62 rows containing missing values (`geom_point()`).
\end{verbatim}

\includegraphics{r4ds_files/figure-latex/unnamed-chunk-14-1.pdf}

Os valores da variável serão representados de modo a atender todas as estéticas simultaneamente, por exemplo, no gráfico acima é dada uma cor, um formato e um tamanho específicos para cada classe de veículo. Os veículos de dois lugares são exibidos como um disco rosa pequeno.

\end{solution}

\hypertarget{exr1-3-5}{%
\subsection*{Exercício 1.3.5}\label{exr1-3-5}}
\addcontentsline{toc}{subsection}{Exercício 1.3.5}

O que a estética \texttt{stroke} faz? com que formas ela trabalha?

\begin{solution}
\leavevmode

\begin{Shaded}
\begin{Highlighting}[]
\FunctionTok{ggplot}\NormalTok{(}\AttributeTok{data =}\NormalTok{ mpg) }\SpecialCharTok{+}
    \FunctionTok{geom\_point}\NormalTok{(}\AttributeTok{mapping =} \FunctionTok{aes}\NormalTok{(}\AttributeTok{x =}\NormalTok{ displ, }\AttributeTok{y =}\NormalTok{ hwy, }\AttributeTok{stroke =}\NormalTok{ displ)) }\SpecialCharTok{+}
\NormalTok{    tema}
\end{Highlighting}
\end{Shaded}

\includegraphics{r4ds_files/figure-latex/unnamed-chunk-15-1.pdf}

A estética \texttt{stroke} controla a espessura do ponto ou elemento a ser representado.

\end{solution}

\hypertarget{exr1-3-6}{%
\subsection*{Exercício 1.3.6}\label{exr1-3-6}}
\addcontentsline{toc}{subsection}{Exercício 1.3.6}

O que acontece se você mapear uma estética a algo diferente de um nome de variável, como \texttt{aes(color\ =\ displ\ \textless{}\ 5)}?

\begin{solution}
\leavevmode

\begin{Shaded}
\begin{Highlighting}[]
\FunctionTok{ggplot}\NormalTok{(}\AttributeTok{data =}\NormalTok{ mpg) }\SpecialCharTok{+}
    \FunctionTok{geom\_point}\NormalTok{(}\AttributeTok{mapping =} \FunctionTok{aes}\NormalTok{(}\AttributeTok{x =}\NormalTok{ displ, }\AttributeTok{y =}\NormalTok{ hwy, }\AttributeTok{color =}\NormalTok{ displ }\SpecialCharTok{\textless{}} \DecValTok{5}\NormalTok{)) }\SpecialCharTok{+}
\NormalTok{    tema}
\end{Highlighting}
\end{Shaded}

\includegraphics{r4ds_files/figure-latex/unnamed-chunk-16-1.pdf}

A expressão é avaliada para cada um dos valores da variável e o resultado é utilizado para plotagem da estética no gráfico.

\end{solution}

\hypertarget{problemas-comuns}{%
\section{Problemas comuns}\label{problemas-comuns}}

Não temos exercícios nessa seção.

\hypertarget{facetas}{%
\section{Facetas}\label{facetas}}

\hypertarget{exr1-5-1}{%
\subsection*{Exercício 1.5.1}\label{exr1-5-1}}
\addcontentsline{toc}{subsection}{Exercício 1.5.1}

O que acontece se você criar facetas em uma variável contínua?

\begin{solution}
\leavevmode

\begin{Shaded}
\begin{Highlighting}[]
\FunctionTok{ggplot}\NormalTok{(}\AttributeTok{data =}\NormalTok{ mpg) }\SpecialCharTok{+}
    \FunctionTok{geom\_point}\NormalTok{(}\AttributeTok{mapping =} \FunctionTok{aes}\NormalTok{(}\AttributeTok{x =}\NormalTok{ displ, }\AttributeTok{y =}\NormalTok{ hwy)) }\SpecialCharTok{+}
    \FunctionTok{facet\_wrap}\NormalTok{(. }\SpecialCharTok{\textasciitilde{}}\NormalTok{ displ) }\SpecialCharTok{+}
\NormalTok{    tema}
\end{Highlighting}
\end{Shaded}

\includegraphics{r4ds_files/figure-latex/unnamed-chunk-17-1.pdf}

O \emph{ggplot} se encarrega de dividir o conjunto em classes e toma o ponto médio de cada classe para realizar a quebra em facetas.

\end{solution}

\hypertarget{exr1-5-2}{%
\subsection*{Exercício 1.5.2}\label{exr1-5-2}}
\addcontentsline{toc}{subsection}{Exercício 1.5.2}

O que significam as célula em branco em um gráfico com \texttt{facet\_grid(drv\ \textasciitilde{}\ cyl)}? Como elas se relacionam a este gráfico?

\begin{Shaded}
\begin{Highlighting}[]
\FunctionTok{ggplot}\NormalTok{(}\AttributeTok{data =}\NormalTok{ mpg) }\SpecialCharTok{+}
    \FunctionTok{geom\_point}\NormalTok{(}\AttributeTok{mapping =} \FunctionTok{aes}\NormalTok{(}\AttributeTok{x =}\NormalTok{ displ, }\AttributeTok{y =}\NormalTok{ hwy)) }\SpecialCharTok{+}
    \FunctionTok{facet\_grid}\NormalTok{(drv }\SpecialCharTok{\textasciitilde{}}\NormalTok{ cyl) }\SpecialCharTok{+}
\NormalTok{    tema}
\end{Highlighting}
\end{Shaded}

\includegraphics{r4ds_files/figure-latex/unnamed-chunk-18-1.pdf}

\begin{solution}
Significa que para aquela combinação de variáveis, não há nenhum valor observado. Por exemplo, não há nenhum veículo com 5 cilindros e tração nas quatro rodas.
\end{solution}

\hypertarget{exr1-5-3}{%
\subsection*{Exercício 1.5.3}\label{exr1-5-3}}
\addcontentsline{toc}{subsection}{Exercício 1.5.3}

Que gráficos o código a seguir faz? O que \texttt{.} faz?

\begin{Shaded}
\begin{Highlighting}[]
\FunctionTok{ggplot}\NormalTok{(}\AttributeTok{data =}\NormalTok{ mpg) }\SpecialCharTok{+}
    \FunctionTok{geom\_point}\NormalTok{(}\AttributeTok{mapping =} \FunctionTok{aes}\NormalTok{(}\AttributeTok{x =}\NormalTok{ displ, }\AttributeTok{y =}\NormalTok{ hwy)) }\SpecialCharTok{+}
    \FunctionTok{facet\_grid}\NormalTok{(drv }\SpecialCharTok{\textasciitilde{}}\NormalTok{ .) }\SpecialCharTok{+}
\NormalTok{    tema}
\end{Highlighting}
\end{Shaded}

\includegraphics{r4ds_files/figure-latex/unnamed-chunk-19-1.pdf}

\begin{Shaded}
\begin{Highlighting}[]
\FunctionTok{ggplot}\NormalTok{(}\AttributeTok{data =}\NormalTok{ mpg) }\SpecialCharTok{+}
    \FunctionTok{geom\_point}\NormalTok{(}\AttributeTok{mapping =} \FunctionTok{aes}\NormalTok{(}\AttributeTok{x =}\NormalTok{ displ, }\AttributeTok{y =}\NormalTok{ hwy)) }\SpecialCharTok{+}
    \FunctionTok{facet\_grid}\NormalTok{(. }\SpecialCharTok{\textasciitilde{}}\NormalTok{ cyl) }\SpecialCharTok{+}
\NormalTok{    tema}
\end{Highlighting}
\end{Shaded}

\includegraphics{r4ds_files/figure-latex/unnamed-chunk-20-1.pdf}

\begin{solution}
São gerados os gráficos de dispersão segregados pelas variáveis \texttt{drv} e \texttt{cyl}, respectivamente. O \texttt{.} indica que não queremos considerar nenhuma segregação naquela dimensão do \emph{grid} (linha ou coluna).
\end{solution}

\hypertarget{exr1-5-4}{%
\subsection*{Exercício 1.5.4}\label{exr1-5-4}}
\addcontentsline{toc}{subsection}{Exercício 1.5.4}

Pegue o primeiro gráfico em facetas dessa seção.

\begin{Shaded}
\begin{Highlighting}[]
\FunctionTok{ggplot}\NormalTok{(}\AttributeTok{data =}\NormalTok{ mpg) }\SpecialCharTok{+}
    \FunctionTok{geom\_point}\NormalTok{(}\AttributeTok{data =} \FunctionTok{transform}\NormalTok{(mpg, }\AttributeTok{class =} \ConstantTok{NULL}\NormalTok{), }\AttributeTok{mapping =} \FunctionTok{aes}\NormalTok{(}\AttributeTok{x =}\NormalTok{ displ, }\AttributeTok{y =}\NormalTok{ hwy), }\AttributeTok{color =} \StringTok{"gray80"}\NormalTok{) }\SpecialCharTok{+}
    \FunctionTok{geom\_point}\NormalTok{(}\AttributeTok{mapping =} \FunctionTok{aes}\NormalTok{(}\AttributeTok{x =}\NormalTok{ displ, }\AttributeTok{y =}\NormalTok{ hwy)) }\SpecialCharTok{+}
    \FunctionTok{facet\_wrap}\NormalTok{(}\SpecialCharTok{\textasciitilde{}}\NormalTok{ class, }\AttributeTok{nrow =} \DecValTok{2}\NormalTok{) }\SpecialCharTok{+}
\NormalTok{    tema}
\end{Highlighting}
\end{Shaded}

\includegraphics{r4ds_files/figure-latex/unnamed-chunk-21-1.pdf}

Quais são as vantagens de usar facetas, em vez de estética de cor? Quais são as desvantagens? Como o equilíbrio poderia mudar se você tivesse um conjunto de dados maior?

\begin{solution}
A principal vantagem no uso de facetas é que fica mais fácil analisar os dados quando eles estão separados em seu próprio contexto, contudo visualizá-los assim dificulta a comparação entre grupos.
\end{solution}

\hypertarget{exr1-5-5}{%
\subsection*{Exercício 1.5.5}\label{exr1-5-5}}
\addcontentsline{toc}{subsection}{Exercício 1.5.5}

Leia \texttt{?facet\_wrap}. O que \texttt{nrow} faz? o que \texttt{ncol} faz? Quais outras opções controlam o layout de paineis individuais? Por que \texttt{facet\_grid()} não tem variáveis \texttt{nrow}e \texttt{ncol}?

\begin{solution}
\leavevmode

\begin{verbatim}
?facet_wrap
\end{verbatim}

Os atributos \texttt{ncol} e \texttt{nrow} são utilizados pelo \texttt{facet\_wrap} para determinar o número de colunas ou linhas (respectivamente) nas quais serão distribuídos os gráficos segregados. Esses atributos não figuram em \texttt{facet\_grid} pelo fato deste já organizar as facetas retangularmente.

\end{solution}

\hypertarget{exr1-5-6}{%
\subsection*{Exercício 1.5.6}\label{exr1-5-6}}
\addcontentsline{toc}{subsection}{Exercício 1.5.6}

Ao usar \texttt{facet\_grid()} você normalmente deveria colocar a variável com níveis mais singulares nas colunas. Por quê?

\begin{solution}
Para melhor aproveitamento do espaço em tela.
\end{solution}

\hypertarget{objetos-geomuxe9tricos}{%
\section{Objetos geométricos}\label{objetos-geomuxe9tricos}}

\hypertarget{exr1-6-1}{%
\subsection*{Exercício 1.6.1}\label{exr1-6-1}}
\addcontentsline{toc}{subsection}{Exercício 1.6.1}

Que \emph{geom} você usaria para desenhar um gráfico de linha? Um diagrama de caixas (\emph{boxplot})? Um histograma? Um gráfico de área?

\begin{solution}
\leavevmode

\begin{Shaded}
\begin{Highlighting}[]
\FunctionTok{ggplot}\NormalTok{(}\AttributeTok{data =}\NormalTok{ mpg, }\AttributeTok{mapping =} \FunctionTok{aes}\NormalTok{(}\AttributeTok{x =}\NormalTok{ displ, }\AttributeTok{y =}\NormalTok{ hwy)) }\SpecialCharTok{+}
    \FunctionTok{geom\_line}\NormalTok{() }\SpecialCharTok{+}
\NormalTok{    tema}
\end{Highlighting}
\end{Shaded}

\includegraphics{r4ds_files/figure-latex/unnamed-chunk-22-1.pdf}

\begin{Shaded}
\begin{Highlighting}[]
\FunctionTok{ggplot}\NormalTok{(}\AttributeTok{data =}\NormalTok{ mpg) }\SpecialCharTok{+}
    \FunctionTok{geom\_boxplot}\NormalTok{(}\AttributeTok{mapping =} \FunctionTok{aes}\NormalTok{(}\AttributeTok{y =}\NormalTok{ hwy, }\AttributeTok{x =}\NormalTok{ class)) }\SpecialCharTok{+}
\NormalTok{    tema}
\end{Highlighting}
\end{Shaded}

\includegraphics{r4ds_files/figure-latex/unnamed-chunk-23-1.pdf}

\begin{Shaded}
\begin{Highlighting}[]
\FunctionTok{ggplot}\NormalTok{(}\AttributeTok{data =}\NormalTok{ mpg, }\AttributeTok{mapping =} \FunctionTok{aes}\NormalTok{(}\AttributeTok{x =}\NormalTok{ hwy)) }\SpecialCharTok{+}
    \FunctionTok{geom\_histogram}\NormalTok{() }\SpecialCharTok{+}
\NormalTok{    tema}
\end{Highlighting}
\end{Shaded}

\begin{verbatim}
## `stat_bin()` using `bins = 30`. Pick better value with `binwidth`.
\end{verbatim}

\includegraphics{r4ds_files/figure-latex/unnamed-chunk-24-1.pdf}

\begin{Shaded}
\begin{Highlighting}[]
\FunctionTok{ggplot}\NormalTok{(}\AttributeTok{data =}\NormalTok{ economics, }\AttributeTok{mapping =} \FunctionTok{aes}\NormalTok{(}\AttributeTok{x =}\NormalTok{ date, }\AttributeTok{y =}\NormalTok{ unemploy)) }\SpecialCharTok{+}
    \FunctionTok{geom\_area}\NormalTok{() }\SpecialCharTok{+}
\NormalTok{    tema}
\end{Highlighting}
\end{Shaded}

\includegraphics{r4ds_files/figure-latex/unnamed-chunk-25-1.pdf}

Podem ser utilizados, respectivamente as \emph{geoms}: \emph{line}, \emph{boxplot}, \emph{histogram} e \emph{area}.

\end{solution}

\hypertarget{exr1-6-2}{%
\subsection*{Exercício 1.6.2}\label{exr1-6-2}}
\addcontentsline{toc}{subsection}{Exercício 1.6.2}

Execute este código em sua cabeça e preveja como será o resultado. Depois execute o código no R e confira suas previsões:

\begin{Shaded}
\begin{Highlighting}[]
\FunctionTok{ggplot}\NormalTok{(}\AttributeTok{data =}\NormalTok{ mpg, }\AttributeTok{mapping =} \FunctionTok{aes}\NormalTok{(}\AttributeTok{x =}\NormalTok{ displ, }\AttributeTok{y =}\NormalTok{ hwy, }\AttributeTok{color =}\NormalTok{ drv)) }\SpecialCharTok{+}
    \FunctionTok{geom\_point}\NormalTok{() }\SpecialCharTok{+}
    \FunctionTok{geom\_smooth}\NormalTok{(}\AttributeTok{se =} \ConstantTok{FALSE}\NormalTok{) }\SpecialCharTok{+}
\NormalTok{    tema}
\end{Highlighting}
\end{Shaded}

\begin{verbatim}
## `geom_smooth()` using method = 'loess' and formula = 'y ~ x'
\end{verbatim}

\includegraphics{r4ds_files/figure-latex/unnamed-chunk-26-1.pdf}

\begin{solution}
O gráfico bateu com a expectativa.
\end{solution}

\hypertarget{exr1-6-3}{%
\subsection*{Exercício 1.6.3}\label{exr1-6-3}}
\addcontentsline{toc}{subsection}{Exercício 1.6.3}

O que o \texttt{show.legend\ =\ FALSE} faz? O que acontece se você removê-lo? Por que você acha que usei isso anteriormente no capítulo?

\begin{solution}
\leavevmode

\begin{Shaded}
\begin{Highlighting}[]
\FunctionTok{ggplot}\NormalTok{(}\AttributeTok{data =}\NormalTok{ mpg, }\AttributeTok{mapping =} \FunctionTok{aes}\NormalTok{(}\AttributeTok{x =}\NormalTok{ displ, }\AttributeTok{y =}\NormalTok{ hwy, }\AttributeTok{color =}\NormalTok{ drv)) }\SpecialCharTok{+}
    \FunctionTok{geom\_point}\NormalTok{(}\AttributeTok{show.legend =} \ConstantTok{FALSE}\NormalTok{) }\SpecialCharTok{+}
    \FunctionTok{geom\_smooth}\NormalTok{(}\AttributeTok{se =} \ConstantTok{FALSE}\NormalTok{, }\AttributeTok{show.legend =} \ConstantTok{FALSE}\NormalTok{) }\SpecialCharTok{+}
\NormalTok{    tema}
\end{Highlighting}
\end{Shaded}

\begin{verbatim}
## `geom_smooth()` using method = 'loess' and formula = 'y ~ x'
\end{verbatim}

\includegraphics{r4ds_files/figure-latex/unnamed-chunk-27-1.pdf}

Ele indica que, para a camada à qual se aplica, não serão geradas as legendas de identificação.

\end{solution}

\hypertarget{exr1-6-4}{%
\subsection*{Exercício 1.6.4}\label{exr1-6-4}}
\addcontentsline{toc}{subsection}{Exercício 1.6.4}

O que o argumento \texttt{se} para \texttt{geom\_smooth} faz?

\begin{solution}
\leavevmode

\begin{verbatim}
?geom_smooth
\end{verbatim}

Esse argumento indica se o intervalo de confiança utilizado no processo de suavização da linha deve ou não ser exibido no gráfico.

\end{solution}

\hypertarget{exr1-6-5}{%
\subsection*{Exercício 1.6.5}\label{exr1-6-5}}
\addcontentsline{toc}{subsection}{Exercício 1.6.5}

Esses dois gráficos serão diferentes? Por quê/por que não?

\begin{verbatim}
ggplot(data = mpg, mapping = aes(x = displ, y = hwy)) +
    geom_point() +
    geom_smooth() +
    tema
    
ggplot() + 
    geom_point(data = mpg, mapping = aes(x = displ, y = hwy)) +
    geom_smooth(data = mpg, mapping = aes(x = displ, y = hwy)) +
    tema
\end{verbatim}

\begin{solution}
Os gráficos serão iguais. Ao informar os parâmetros \texttt{data} e \texttt{mapping} na função \texttt{ggplot} essas atributos serão considerados como globais, sendo utilizado em todos as camadas do gráfico, a menos que alguma das camadas os sobrescreva. No segundo gráfico, não são definidos parâmetros globais, porém, o mesmo parâmetro é passado para ambas as camadas, sendo assim, a única diferença é o código estar duplicado.
\end{solution}

\hypertarget{exr1-6-6}{%
\subsection*{Exercício 1.6.6}\label{exr1-6-6}}
\addcontentsline{toc}{subsection}{Exercício 1.6.6}

Recrie o código R necessário para gerar os seguintes gráficos:

\includegraphics{r4ds_files/figure-latex/unnamed-chunk-28-1.pdf}

\begin{solution}
\leavevmode

\begin{verbatim}
a <- ggplot(data = mpg, mapping = aes(x = displ, y = hwy)) +
        geom_point() +
        geom_smooth(se = FALSE) +
        tema

b <- ggplot(data = mpg, mapping = aes(x = displ, y = hwy)) +
        geom_point() +
        geom_smooth(mapping = aes(group = drv), se = FALSE) +
        tema

c <- ggplot(data = mpg, mapping = aes(x = displ, y = hwy, color = drv)) +
        geom_point() +
        geom_smooth(se = FALSE) +
        tema

d <- ggplot(data = mpg, mapping = aes(x = displ, y = hwy)) +
        geom_point(mapping = aes(color = drv)) +
        geom_smooth(se = FALSE) +
        tema

e <- ggplot(data = mpg, mapping = aes(x = displ, y = hwy)) +
        geom_point(mapping = aes(color = drv)) +
        geom_smooth(mapping = aes(linetype = drv), se = FALSE) +
        tema

f <- ggplot(data = mpg, mapping = aes(x = displ, y = hwy, fill = drv)) +
        geom_point(color = "white", shape = 21, size = 3, stroke = 2) +
        tema
\end{verbatim}

\end{solution}

\hypertarget{transformauxe7uxf5es-estatuxedsticas}{%
\section{Transformações estatísticas}\label{transformauxe7uxf5es-estatuxedsticas}}

\hypertarget{exr1-7-1}{%
\subsection*{Exercício 1.7.1}\label{exr1-7-1}}
\addcontentsline{toc}{subsection}{Exercício 1.7.1}

Qual é o \texttt{geom} padrão associado ao \texttt{stat\_summary()}? Como você poderia reescrever o gráfico anterior usando essa função \texttt{geom}, em vez da função \texttt{stat}?

\begin{solution}
\leavevmode

\begin{verbatim}
?stat_summary
\end{verbatim}

\begin{Shaded}
\begin{Highlighting}[]
\FunctionTok{ggplot}\NormalTok{(}\AttributeTok{data =}\NormalTok{ diamonds) }\SpecialCharTok{+}
    \FunctionTok{stat\_summary}\NormalTok{(}
        \AttributeTok{mapping =} \FunctionTok{aes}\NormalTok{(}\AttributeTok{x =}\NormalTok{ cut, }\AttributeTok{y =}\NormalTok{ depth),}
        \AttributeTok{fun.min =}\NormalTok{ min,}
        \AttributeTok{fun.max =}\NormalTok{ max,}
        \AttributeTok{fun =}\NormalTok{ median}
\NormalTok{    ) }\SpecialCharTok{+}
\NormalTok{    tema}
\end{Highlighting}
\end{Shaded}

\includegraphics{r4ds_files/figure-latex/unnamed-chunk-29-1.pdf}

A \texttt{geom} associada é a \texttt{geom\_pointrange} e o gráfico poderia ser reescrito da seguinte maneira.

\end{solution}

\hypertarget{exr1-7-2}{%
\subsection*{Exercício 1.7.2}\label{exr1-7-2}}
\addcontentsline{toc}{subsection}{Exercício 1.7.2}

O que \texttt{geom\_col()} faz? Qual é a diferença entre ele e \texttt{geom\_bar()}?

\begin{solution}
\leavevmode

\begin{Shaded}
\begin{Highlighting}[]
\FunctionTok{ggplot}\NormalTok{(}\AttributeTok{data =}\NormalTok{ diamonds, }\AttributeTok{mapping =} \FunctionTok{aes}\NormalTok{(}\AttributeTok{x =}\NormalTok{ cut)) }\SpecialCharTok{+}
    \FunctionTok{geom\_bar}\NormalTok{() }\SpecialCharTok{+}
\NormalTok{    tema}
\end{Highlighting}
\end{Shaded}

\includegraphics{r4ds_files/figure-latex/unnamed-chunk-30-1.pdf}

\begin{Shaded}
\begin{Highlighting}[]
\FunctionTok{ggplot}\NormalTok{(}\AttributeTok{data =}\NormalTok{ diamonds, }\AttributeTok{mapping =} \FunctionTok{aes}\NormalTok{(}\AttributeTok{x =}\NormalTok{ cut, }\AttributeTok{y =}\NormalTok{ carat)) }\SpecialCharTok{+}
    \FunctionTok{geom\_col}\NormalTok{() }\SpecialCharTok{+}
\NormalTok{    tema}
\end{Highlighting}
\end{Shaded}

\includegraphics{r4ds_files/figure-latex/unnamed-chunk-31-1.pdf}

Enquanto no \texttt{geom\_bar} a altura das barras representa uma transformação estatística relacionada às observações (como \texttt{count}, por exemplo), no \texttt{geom\_col} podemos exibir o acumulado (soma) de uma variável para cada categoria exibida.

\end{solution}

\hypertarget{exr1-7-3}{%
\subsection*{Exercício 1.7.3}\label{exr1-7-3}}
\addcontentsline{toc}{subsection}{Exercício 1.7.3}

A maioria dos \texttt{geoms} e \texttt{stats} vem em pares, que são quase sempre usados juntos. Leia a documentação e faça uma lista de todos os pares. O que eles têm em comum?

\begin{solution}
\leavevmode

\begin{longtable}[]{@{}ccc@{}}
\toprule\noalign{}
\# & Geom & Stat \\
\midrule\noalign{}
\endhead
\bottomrule\noalign{}
\endlastfoot
01 & Blank & Identity \\
02 & Curve & Identity \\
03 & Segment & Identity \\
04 & Path & Identity \\
05 & Line & Identity \\
06 & Step & Identity \\
07 & Poligon & Identity \\
08 & Raster & Identity \\
09 & Rect & Identity \\
10 & Tile & Identity \\
11 & Ribbon & Identity \\
12 & Area & Identity \\
13 & Align & ? \\
14 & ABLine & ? \\
15 & HLine & ? \\
16 & Density & Density \\
17 & DotPlot & ? \\
18 & Freqpoly & Bin \\
19 & Histogram & Bin \\
20 & Col & Identity \\
21 & Bar & Count \\
22 & Label & Identity \\
23 & Text & Identity \\
24 & Jitter & Identity \\
25 & Point & Identity \\
26 & Quantile & Quantile \\
27 & Rug & Identity \\
28 & Boxplot & Boxplot \\
29 & Violin & YDensity \\
30 & Count & Sum \\
31 & Bin 2D & Bin 2D \\
32 & Density 2D & Density 2D \\
33 & Hex & Bin Hex \\
34 & Cross Bar & Identity \\
35 & Error Bar & Identity \\
36 & Line Range & Identity \\
37 & Point Range & Identity \\
38 & Map & Identity \\
39 & Contour & Contour \\
40 & Contour Filled & Contour Filled \\
\end{longtable}

\end{solution}

\hypertarget{exr1-7-4}{%
\subsection*{Exercício 1.7.4}\label{exr1-7-4}}
\addcontentsline{toc}{subsection}{Exercício 1.7.4}

Quais variáveis \texttt{stat\_smooth()} calcula? Quais parâmetros controlam seu comportamento?

\begin{solution}
\leavevmode

\begin{verbatim}
?stat_smooth
\end{verbatim}

\end{solution}

\hypertarget{exr1-7-5}{%
\subsection*{Exercício 1.7.5}\label{exr1-7-5}}
\addcontentsline{toc}{subsection}{Exercício 1.7.5}

Em nosso gráfico de barra de \emph{proportion}, precisamos configurar \texttt{group\ =\ 1}. Por quê? Em outras palavras, qual é o problema com esses dois gráficos?

\begin{Shaded}
\begin{Highlighting}[]
\FunctionTok{ggplot}\NormalTok{(}\AttributeTok{data =}\NormalTok{ diamonds) }\SpecialCharTok{+}
    \FunctionTok{geom\_bar}\NormalTok{(}\AttributeTok{mapping =} \FunctionTok{aes}\NormalTok{(}\AttributeTok{x =}\NormalTok{ cut, }\AttributeTok{y =} \FunctionTok{after\_stat}\NormalTok{(prop), }\AttributeTok{group =} \DecValTok{1}\NormalTok{)) }\SpecialCharTok{+}
\NormalTok{    tema}
\end{Highlighting}
\end{Shaded}

\includegraphics{r4ds_files/figure-latex/unnamed-chunk-32-1.pdf}

\begin{solution}
\leavevmode

\begin{Shaded}
\begin{Highlighting}[]
\FunctionTok{ggplot}\NormalTok{(}\AttributeTok{data =}\NormalTok{ diamonds) }\SpecialCharTok{+}
    \FunctionTok{geom\_bar}\NormalTok{(}\AttributeTok{mapping =} \FunctionTok{aes}\NormalTok{(}
        \AttributeTok{x =}\NormalTok{ cut, }
        \AttributeTok{fill =}\NormalTok{ color, }
        \AttributeTok{y =} \FunctionTok{after\_stat}\NormalTok{(prop), }
        \AttributeTok{group =}\NormalTok{ color}
\NormalTok{    )) }\SpecialCharTok{+}
\NormalTok{    tema}
\end{Highlighting}
\end{Shaded}

\includegraphics{r4ds_files/figure-latex/unnamed-chunk-33-1.pdf}

Quando estamos trabalhando com proporções (ou estátisticas em geral), é importante destacar para o \texttt{ggplot} qual agrupamento ele deve considerar, caso contrário ele irá considerar um único grupo e dará uma impressão incorreta ao gráfico. No primeiro exemplo, foi utilizado \texttt{group\ =\ 1} (e, na verdade, poderia ser qualquer valor) apenas para indicar que deveria ser realizado um agrupamento.

\end{solution}

\hypertarget{ajustes-de-posiuxe7uxe3o}{%
\section{Ajustes de posição}\label{ajustes-de-posiuxe7uxe3o}}

\hypertarget{exr1-8-1}{%
\subsection*{Exercício 1.8.1}\label{exr1-8-1}}
\addcontentsline{toc}{subsection}{Exercício 1.8.1}

Qual é o problema com este gráfico? Como você poderia melhorá-lo?

\begin{Shaded}
\begin{Highlighting}[]
\FunctionTok{ggplot}\NormalTok{(}\AttributeTok{data =}\NormalTok{ mpg, }\AttributeTok{mapping =} \FunctionTok{aes}\NormalTok{(}\AttributeTok{x =}\NormalTok{ cty, }\AttributeTok{y =}\NormalTok{ hwy)) }\SpecialCharTok{+}
    \FunctionTok{geom\_point}\NormalTok{() }\SpecialCharTok{+}
\NormalTok{    tema}
\end{Highlighting}
\end{Shaded}

\includegraphics{r4ds_files/figure-latex/unnamed-chunk-34-1.pdf}

\begin{solution}
Há pontos sobrepostos. Uma melhoria poderia ser usar \texttt{geom\_jitter} em lugar de \texttt{geom\_point}.

\begin{Shaded}
\begin{Highlighting}[]
\FunctionTok{ggplot}\NormalTok{(}\AttributeTok{data =}\NormalTok{ mpg, }\AttributeTok{mapping =} \FunctionTok{aes}\NormalTok{(}\AttributeTok{x =}\NormalTok{ cty, }\AttributeTok{y =}\NormalTok{ hwy)) }\SpecialCharTok{+}
    \FunctionTok{geom\_jitter}\NormalTok{() }\SpecialCharTok{+}
\NormalTok{    tema}
\end{Highlighting}
\end{Shaded}

\includegraphics{r4ds_files/figure-latex/unnamed-chunk-35-1.pdf}
\end{solution}

\hypertarget{exr1-8-2}{%
\subsection*{Exercício 1.8.2}\label{exr1-8-2}}
\addcontentsline{toc}{subsection}{Exercício 1.8.2}

Quais parâmetros para \texttt{geom\_jitter} controlam a quantidade de oscilação?

\begin{solution}
Conforme a documentação disposta em \texttt{?geom\_jitter}, são utilizados os parâmetros \texttt{width} e \texttt{height}.
\end{solution}

\hypertarget{exr1-8-3}{%
\subsection*{Exercício 1.8.3}\label{exr1-8-3}}
\addcontentsline{toc}{subsection}{Exercício 1.8.3}

Compare o contraste entre \texttt{geom\_jitter} e \texttt{geom\_count}.

\begin{solution}
\leavevmode

\begin{Shaded}
\begin{Highlighting}[]
\NormalTok{a }\OtherTok{\textless{}{-}} \FunctionTok{ggplot}\NormalTok{(}\AttributeTok{data =}\NormalTok{ mpg, }\AttributeTok{mapping =} \FunctionTok{aes}\NormalTok{(}\AttributeTok{x =}\NormalTok{ cty, }\AttributeTok{y =}\NormalTok{ hwy)) }\SpecialCharTok{+}
      \FunctionTok{geom\_jitter}\NormalTok{() }\SpecialCharTok{+}
\NormalTok{      tema}

\NormalTok{b }\OtherTok{\textless{}{-}} \FunctionTok{ggplot}\NormalTok{(}\AttributeTok{data =}\NormalTok{ mpg, }\AttributeTok{mapping =} \FunctionTok{aes}\NormalTok{(}\AttributeTok{x =}\NormalTok{ cty, }\AttributeTok{y =}\NormalTok{ hwy)) }\SpecialCharTok{+}
      \FunctionTok{geom\_count}\NormalTok{(}\AttributeTok{show.legend =} \ConstantTok{FALSE}\NormalTok{) }\SpecialCharTok{+}
\NormalTok{      tema}

\FunctionTok{grid.arrange}\NormalTok{(a, b, }\AttributeTok{nrow =} \DecValTok{2}\NormalTok{)}
\end{Highlighting}
\end{Shaded}

\includegraphics{r4ds_files/figure-latex/unnamed-chunk-36-1.pdf}

Para contornar o problema da sobreposição de pontos, \texttt{geom\_jitter} adiciona um pequeno ruído aleatório aos dados, enquanto o \texttt{geom\_count} contabiliza os pontos sobrepostos e altera o tamanho dos pontos conforme a quantidade.

\end{solution}

\hypertarget{exr1-8-4}{%
\subsection*{Exercício 1.8.4}\label{exr1-8-4}}
\addcontentsline{toc}{subsection}{Exercício 1.8.4}

Qual é o ajuste de posição padrão para \texttt{geom\_boxplot()}? Crie uma visualização do conjunto de dados \texttt{mpg} que demonstre isso.

\begin{solution}
Conforme pode ser visto em \texttt{?geom\_boxplot}, a \texttt{position} padrão é a \texttt{dodge2}.

\begin{Shaded}
\begin{Highlighting}[]
\FunctionTok{ggplot}\NormalTok{(}\AttributeTok{data =}\NormalTok{ mpg, }\AttributeTok{mapping =} \FunctionTok{aes}\NormalTok{(}\AttributeTok{x =}\NormalTok{ class, }\AttributeTok{y =}\NormalTok{ hwy)) }\SpecialCharTok{+}
    \FunctionTok{geom\_boxplot}\NormalTok{() }\SpecialCharTok{+}
\NormalTok{    tema}
\end{Highlighting}
\end{Shaded}

\includegraphics{r4ds_files/figure-latex/unnamed-chunk-37-1.pdf}
\end{solution}

\hypertarget{sistemas-de-coordenadas}{%
\section{Sistemas de coordenadas}\label{sistemas-de-coordenadas}}

\hypertarget{exr1-9-1}{%
\subsection*{Exercício 1.9.1}\label{exr1-9-1}}
\addcontentsline{toc}{subsection}{Exercício 1.9.1}

Transforme um gráfico de barras empilhadas em um gráfico de pizza usando \texttt{coord\_polar()}.

\begin{solution}
\leavevmode

\begin{Shaded}
\begin{Highlighting}[]
\FunctionTok{ggplot}\NormalTok{(}\AttributeTok{data =}\NormalTok{ diamonds, }\AttributeTok{mapping =} \FunctionTok{aes}\NormalTok{(}\AttributeTok{x =}\NormalTok{ cut, }\AttributeTok{fill =}\NormalTok{ cut)) }\SpecialCharTok{+}
    \FunctionTok{geom\_bar}\NormalTok{(}\AttributeTok{show.legend =} \ConstantTok{FALSE}\NormalTok{, }\AttributeTok{width =} \DecValTok{1}\NormalTok{) }\SpecialCharTok{+}
    \FunctionTok{coord\_polar}\NormalTok{() }\SpecialCharTok{+}
    \FunctionTok{labs}\NormalTok{(}\AttributeTok{x =} \ConstantTok{NULL}\NormalTok{, }\AttributeTok{y =} \ConstantTok{NULL}\NormalTok{) }\SpecialCharTok{+}
    \FunctionTok{theme}\NormalTok{(}\AttributeTok{aspect.ratio =} \DecValTok{1}\NormalTok{) }\SpecialCharTok{+}
\NormalTok{    tema}
\end{Highlighting}
\end{Shaded}

\includegraphics{r4ds_files/figure-latex/unnamed-chunk-38-1.pdf}

\end{solution}

\hypertarget{exr1-9-2}{%
\subsection*{Exercício 1.9.2}\label{exr1-9-2}}
\addcontentsline{toc}{subsection}{Exercício 1.9.2}

O que \texttt{labs()} faz? Leia a documentação.

\begin{solution}
Usando o comando \texttt{?labs}, vimos que esta função é utilizada para definir labels do gráfico, como título, subtítulo, títulos de eixos, etc.
\end{solution}

\hypertarget{exr1-9-3}{%
\subsection*{Exercício 1.9.3}\label{exr1-9-3}}
\addcontentsline{toc}{subsection}{Exercício 1.9.3}

Qual é a diferença entre \texttt{coord\_quickmap()} e \texttt{coord\_map()}?

\begin{solution}
Usando o comando \texttt{?coord\_map}, notamos que a diferença é que enquanto \texttt{coord\_map()} não preserva linhas retas, sendo assim, mais custoso computacionalmente, o \texttt{coord\_quickmap()} o faz.
\end{solution}

\hypertarget{exr1-9-4}{%
\subsection*{Exercício 1.9.4}\label{exr1-9-4}}
\addcontentsline{toc}{subsection}{Exercício 1.9.4}

O que o gráfico a seguir lhe diz sobre a relação entre \texttt{mpg} de cidade e estrada? Por que \texttt{coord\_fixed()} é importante? O que \texttt{geom\_abline()} faz?

\begin{Shaded}
\begin{Highlighting}[]
\FunctionTok{ggplot}\NormalTok{(}\AttributeTok{data =}\NormalTok{ mpg, }\AttributeTok{mapping =} \FunctionTok{aes}\NormalTok{(}\AttributeTok{x =}\NormalTok{ cty, }\AttributeTok{y =}\NormalTok{ hwy)) }\SpecialCharTok{+}
    \FunctionTok{geom\_point}\NormalTok{() }\SpecialCharTok{+}
    \FunctionTok{geom\_abline}\NormalTok{() }\SpecialCharTok{+}
    \FunctionTok{coord\_fixed}\NormalTok{(}\AttributeTok{ratio =} \DecValTok{1}\NormalTok{, }\AttributeTok{xlim =} \FunctionTok{c}\NormalTok{(}\DecValTok{5}\NormalTok{, }\DecValTok{45}\NormalTok{), }\AttributeTok{ylim =} \FunctionTok{c}\NormalTok{(}\DecValTok{5}\NormalTok{, }\DecValTok{45}\NormalTok{)) }\SpecialCharTok{+}
\NormalTok{    tema}
\end{Highlighting}
\end{Shaded}

\includegraphics{r4ds_files/figure-latex/unnamed-chunk-39-1.pdf}

\begin{solution}
O gráfico mostra a relação entre a eficiência na cidade e na estrada. O \texttt{coord\_fixed()} força que seja mantida uma proporção entre os eixos x e y, isto é, garante que uma unidade no eixo y corresponda a um número determinado de unidades no eixo x. A razão padrão é 1. Já o \texttt{geom\_abline()} define uma linha de referência diagonal ao gráfico, no nosso caso, a linha é a reta dada por \(y - x = 0\).
\end{solution}

\hypertarget{a-gramuxe1tica-em-camadas-de-gruxe1ficos}{%
\section{A gramática em camadas de gráficos}\label{a-gramuxe1tica-em-camadas-de-gruxe1ficos}}

Não temos exercícios nesta seção.

\hypertarget{fluxo-de-trabalho-o-buxe1sico}{%
\chapter{Fluxo de trabalho: o básico}\label{fluxo-de-trabalho-o-buxe1sico}}

\hypertarget{o-buxe1sico-de-programauxe7uxe3o}{%
\section{O básico de programação}\label{o-buxe1sico-de-programauxe7uxe3o}}

Não temos exercícios nesta seção.

\hypertarget{o-que-huxe1-em-um-nome}{%
\section{O que há em um nome?}\label{o-que-huxe1-em-um-nome}}

Não temos exercícios nesta seção.

\hypertarget{chamando-funuxe7uxf5es}{%
\section{Chamando funções}\label{chamando-funuxe7uxf5es}}

\hypertarget{exr2-3-1}{%
\subsection*{Exercício 2.3.1}\label{exr2-3-1}}
\addcontentsline{toc}{subsection}{Exercício 2.3.1}

Por que esse código não funciona?

\begin{verbatim}
my_variable <- 10
my_varIable
\end{verbatim}

\begin{solution}
Foi atribuído um valor à variável \texttt{my\_variable}, contudo depois tentou-se utilizar essa variável, porém a escrita está incorreta e o \texttt{R} não reconheceu a variável.
O \texttt{R} diferencia letras maiúsculas e minúsculas, isto é, as variáveis \texttt{my\_variable} e \texttt{my\_varIable} são distintas.
\end{solution}

\hypertarget{exr2-3-2}{%
\subsection*{Exercício 2.3.2}\label{exr2-3-2}}
\addcontentsline{toc}{subsection}{Exercício 2.3.2}

Ajuste cada um dos seguintes comandos de \texttt{R} para que executem corretamente.

\begin{verbatim}
library(tidyverse)

ggplot(dota = mpg) +
    geom_point(mapping = aes(x = displ, y = hwy))
    
filter(mpg, cyl = 8)
filter(diamond, carat > 3)
\end{verbatim}

\begin{solution}
\leavevmode

\begin{Shaded}
\begin{Highlighting}[]
\FunctionTok{library}\NormalTok{(tidyverse)}

\FunctionTok{ggplot}\NormalTok{(}\AttributeTok{data =}\NormalTok{ mpg) }\SpecialCharTok{+}
    \FunctionTok{geom\_point}\NormalTok{(}\AttributeTok{mapping =} \FunctionTok{aes}\NormalTok{(}\AttributeTok{x =}\NormalTok{ displ, }\AttributeTok{y =}\NormalTok{ hwy))}
\end{Highlighting}
\end{Shaded}

\includegraphics{r4ds_files/figure-latex/unnamed-chunk-40-1.pdf}

\begin{Shaded}
\begin{Highlighting}[]
\FunctionTok{filter}\NormalTok{(mpg, cyl }\SpecialCharTok{==} \DecValTok{8}\NormalTok{)}
\end{Highlighting}
\end{Shaded}

\begin{verbatim}
## # A tibble: 70 x 11
##    manufacturer model      displ  year   cyl trans drv     cty   hwy fl    class
##    <chr>        <chr>      <dbl> <int> <int> <chr> <chr> <int> <int> <chr> <chr>
##  1 audi         a6 quattro   4.2  2008     8 auto~ 4        16    23 p     mids~
##  2 chevrolet    c1500 sub~   5.3  2008     8 auto~ r        14    20 r     suv  
##  3 chevrolet    c1500 sub~   5.3  2008     8 auto~ r        11    15 e     suv  
##  4 chevrolet    c1500 sub~   5.3  2008     8 auto~ r        14    20 r     suv  
##  5 chevrolet    c1500 sub~   5.7  1999     8 auto~ r        13    17 r     suv  
##  6 chevrolet    c1500 sub~   6    2008     8 auto~ r        12    17 r     suv  
##  7 chevrolet    corvette     5.7  1999     8 manu~ r        16    26 p     2sea~
##  8 chevrolet    corvette     5.7  1999     8 auto~ r        15    23 p     2sea~
##  9 chevrolet    corvette     6.2  2008     8 manu~ r        16    26 p     2sea~
## 10 chevrolet    corvette     6.2  2008     8 auto~ r        15    25 p     2sea~
## # i 60 more rows
\end{verbatim}

\begin{Shaded}
\begin{Highlighting}[]
\FunctionTok{filter}\NormalTok{(diamonds, carat }\SpecialCharTok{\textgreater{}} \DecValTok{3}\NormalTok{)}
\end{Highlighting}
\end{Shaded}

\begin{verbatim}
## # A tibble: 32 x 10
##    carat cut     color clarity depth table price     x     y     z
##    <dbl> <ord>   <ord> <ord>   <dbl> <dbl> <int> <dbl> <dbl> <dbl>
##  1  3.01 Premium I     I1       62.7    58  8040  9.1   8.97  5.67
##  2  3.11 Fair    J     I1       65.9    57  9823  9.15  9.02  5.98
##  3  3.01 Premium F     I1       62.2    56  9925  9.24  9.13  5.73
##  4  3.05 Premium E     I1       60.9    58 10453  9.26  9.25  5.66
##  5  3.02 Fair    I     I1       65.2    56 10577  9.11  9.02  5.91
##  6  3.01 Fair    H     I1       56.1    62 10761  9.54  9.38  5.31
##  7  3.65 Fair    H     I1       67.1    53 11668  9.53  9.48  6.38
##  8  3.24 Premium H     I1       62.1    58 12300  9.44  9.4   5.85
##  9  3.22 Ideal   I     I1       62.6    55 12545  9.49  9.42  5.92
## 10  3.5  Ideal   H     I1       62.8    57 12587  9.65  9.59  6.03
## # i 22 more rows
\end{verbatim}

\end{solution}

\hypertarget{exr2-3-3}{%
\subsection*{Exercício 2.3.3}\label{exr2-3-3}}
\addcontentsline{toc}{subsection}{Exercício 2.3.3}

Pressione Alt-Shift-K. O que acontece? Como você pode chegar ao mesmo resultado usando os menus?

\begin{solution}
x
\end{solution}

\hypertarget{transformauxe7uxe3o-de-dados-com-dplyr}{%
\chapter{\texorpdfstring{Transformação de dados com \texttt{dplyr}}{Transformação de dados com dplyr}}\label{transformauxe7uxe3o-de-dados-com-dplyr}}

\hypertarget{introduuxe7uxe3o-1}{%
\section{Introdução}\label{introduuxe7uxe3o-1}}

Não temos exercícios nesta seção.

\hypertarget{filtrar-linhas-com-filter}{%
\section{\texorpdfstring{Filtrar linhas com \texttt{filter()}}{Filtrar linhas com filter()}}\label{filtrar-linhas-com-filter}}

Não temos exercícios nesta seção.

\hypertarget{comparauxe7uxf5es}{%
\section{Comparações}\label{comparauxe7uxf5es}}

\hypertarget{exr3-3-1}{%
\subsection*{Exercício 3.3.1}\label{exr3-3-1}}
\addcontentsline{toc}{subsection}{Exercício 3.3.1}

Encontre todos os voos que:

\begin{enumerate}
\def\labelenumi{\alph{enumi}.}
\tightlist
\item
  Tiveram um atraso de duas horas ou mais na chegada.
\item
  Foram para Houston (IAH ou HOU).
\item
  Foram operados pela United, American ou Delta.
\item
  Partiram em julho, agosto e setembro.
\item
  Chegaram com mais de duas horas de atraso, mas não saíram atrasados.
\item
  Atrasaram pelo menos uma hora, mas compensaram mais de 30 minutos durante o trajeto.
\item
  Saíram entre meia-noite e 6h (incluindo esses horários).
\end{enumerate}

\begin{solution}
\leavevmode

\begin{enumerate}
\def\labelenumi{\alph{enumi}.}
\tightlist
\item
  Tiveram um atraso de duas horas ou mais na chegada.
\end{enumerate}

\begin{Shaded}
\begin{Highlighting}[]
\FunctionTok{filter}\NormalTok{(flights, arr\_delay }\SpecialCharTok{\textgreater{}=} \DecValTok{120}\NormalTok{)}
\end{Highlighting}
\end{Shaded}

\begin{verbatim}
## # A tibble: 10,200 x 19
##     year month   day dep_time sched_dep_time dep_delay arr_time sched_arr_time
##    <int> <int> <int>    <int>          <int>     <dbl>    <int>          <int>
##  1  2013     1     1      811            630       101     1047            830
##  2  2013     1     1      848           1835       853     1001           1950
##  3  2013     1     1      957            733       144     1056            853
##  4  2013     1     1     1114            900       134     1447           1222
##  5  2013     1     1     1505           1310       115     1638           1431
##  6  2013     1     1     1525           1340       105     1831           1626
##  7  2013     1     1     1549           1445        64     1912           1656
##  8  2013     1     1     1558           1359       119     1718           1515
##  9  2013     1     1     1732           1630        62     2028           1825
## 10  2013     1     1     1803           1620       103     2008           1750
## # i 10,190 more rows
## # i 11 more variables: arr_delay <dbl>, carrier <chr>, flight <int>,
## #   tailnum <chr>, origin <chr>, dest <chr>, air_time <dbl>, distance <dbl>,
## #   hour <dbl>, minute <dbl>, time_hour <dttm>
\end{verbatim}

\begin{enumerate}
\def\labelenumi{\alph{enumi}.}
\setcounter{enumi}{1}
\tightlist
\item
  Foram para Houston (IAH ou HOU).
\end{enumerate}

\begin{Shaded}
\begin{Highlighting}[]
\FunctionTok{filter}\NormalTok{(flights, dest }\SpecialCharTok{\%in\%} \FunctionTok{c}\NormalTok{(}\StringTok{"IAH"}\NormalTok{, }\StringTok{"HOU"}\NormalTok{))}
\end{Highlighting}
\end{Shaded}

\begin{verbatim}
## # A tibble: 9,313 x 19
##     year month   day dep_time sched_dep_time dep_delay arr_time sched_arr_time
##    <int> <int> <int>    <int>          <int>     <dbl>    <int>          <int>
##  1  2013     1     1      517            515         2      830            819
##  2  2013     1     1      533            529         4      850            830
##  3  2013     1     1      623            627        -4      933            932
##  4  2013     1     1      728            732        -4     1041           1038
##  5  2013     1     1      739            739         0     1104           1038
##  6  2013     1     1      908            908         0     1228           1219
##  7  2013     1     1     1028           1026         2     1350           1339
##  8  2013     1     1     1044           1045        -1     1352           1351
##  9  2013     1     1     1114            900       134     1447           1222
## 10  2013     1     1     1205           1200         5     1503           1505
## # i 9,303 more rows
## # i 11 more variables: arr_delay <dbl>, carrier <chr>, flight <int>,
## #   tailnum <chr>, origin <chr>, dest <chr>, air_time <dbl>, distance <dbl>,
## #   hour <dbl>, minute <dbl>, time_hour <dttm>
\end{verbatim}

\begin{enumerate}
\def\labelenumi{\alph{enumi}.}
\setcounter{enumi}{2}
\tightlist
\item
  Foram operados pela United, American ou Delta.
\end{enumerate}

\begin{Shaded}
\begin{Highlighting}[]
\FunctionTok{filter}\NormalTok{(flights, carrier }\SpecialCharTok{\%in\%} \FunctionTok{c}\NormalTok{(}\StringTok{"AA"}\NormalTok{, }\StringTok{"DL"}\NormalTok{, }\StringTok{"UA"}\NormalTok{))}
\end{Highlighting}
\end{Shaded}

\begin{verbatim}
## # A tibble: 139,504 x 19
##     year month   day dep_time sched_dep_time dep_delay arr_time sched_arr_time
##    <int> <int> <int>    <int>          <int>     <dbl>    <int>          <int>
##  1  2013     1     1      517            515         2      830            819
##  2  2013     1     1      533            529         4      850            830
##  3  2013     1     1      542            540         2      923            850
##  4  2013     1     1      554            600        -6      812            837
##  5  2013     1     1      554            558        -4      740            728
##  6  2013     1     1      558            600        -2      753            745
##  7  2013     1     1      558            600        -2      924            917
##  8  2013     1     1      558            600        -2      923            937
##  9  2013     1     1      559            600        -1      941            910
## 10  2013     1     1      559            600        -1      854            902
## # i 139,494 more rows
## # i 11 more variables: arr_delay <dbl>, carrier <chr>, flight <int>,
## #   tailnum <chr>, origin <chr>, dest <chr>, air_time <dbl>, distance <dbl>,
## #   hour <dbl>, minute <dbl>, time_hour <dttm>
\end{verbatim}

\begin{enumerate}
\def\labelenumi{\alph{enumi}.}
\setcounter{enumi}{3}
\tightlist
\item
  Partiram em julho, agosto e setembro.
\end{enumerate}

\begin{Shaded}
\begin{Highlighting}[]
\FunctionTok{filter}\NormalTok{(flights, month }\SpecialCharTok{\%in\%} \FunctionTok{c}\NormalTok{(}\DecValTok{7}\NormalTok{, }\DecValTok{8}\NormalTok{, }\DecValTok{9}\NormalTok{))}
\end{Highlighting}
\end{Shaded}

\begin{verbatim}
## # A tibble: 86,326 x 19
##     year month   day dep_time sched_dep_time dep_delay arr_time sched_arr_time
##    <int> <int> <int>    <int>          <int>     <dbl>    <int>          <int>
##  1  2013     7     1        1           2029       212      236           2359
##  2  2013     7     1        2           2359         3      344            344
##  3  2013     7     1       29           2245       104      151              1
##  4  2013     7     1       43           2130       193      322             14
##  5  2013     7     1       44           2150       174      300            100
##  6  2013     7     1       46           2051       235      304           2358
##  7  2013     7     1       48           2001       287      308           2305
##  8  2013     7     1       58           2155       183      335             43
##  9  2013     7     1      100           2146       194      327             30
## 10  2013     7     1      100           2245       135      337            135
## # i 86,316 more rows
## # i 11 more variables: arr_delay <dbl>, carrier <chr>, flight <int>,
## #   tailnum <chr>, origin <chr>, dest <chr>, air_time <dbl>, distance <dbl>,
## #   hour <dbl>, minute <dbl>, time_hour <dttm>
\end{verbatim}

\begin{enumerate}
\def\labelenumi{\alph{enumi}.}
\setcounter{enumi}{4}
\tightlist
\item
  Chegaram com mais de duas horas de atraso, mas não saíram atrasados.
\end{enumerate}

\begin{Shaded}
\begin{Highlighting}[]
\FunctionTok{filter}\NormalTok{(flights, dep\_delay }\SpecialCharTok{\textless{}=} \DecValTok{0}\NormalTok{, arr\_delay }\SpecialCharTok{\textgreater{}} \DecValTok{120}\NormalTok{)}
\end{Highlighting}
\end{Shaded}

\begin{verbatim}
## # A tibble: 29 x 19
##     year month   day dep_time sched_dep_time dep_delay arr_time sched_arr_time
##    <int> <int> <int>    <int>          <int>     <dbl>    <int>          <int>
##  1  2013     1    27     1419           1420        -1     1754           1550
##  2  2013    10     7     1350           1350         0     1736           1526
##  3  2013    10     7     1357           1359        -2     1858           1654
##  4  2013    10    16      657            700        -3     1258           1056
##  5  2013    11     1      658            700        -2     1329           1015
##  6  2013     3    18     1844           1847        -3       39           2219
##  7  2013     4    17     1635           1640        -5     2049           1845
##  8  2013     4    18      558            600        -2     1149            850
##  9  2013     4    18      655            700        -5     1213            950
## 10  2013     5    22     1827           1830        -3     2217           2010
## # i 19 more rows
## # i 11 more variables: arr_delay <dbl>, carrier <chr>, flight <int>,
## #   tailnum <chr>, origin <chr>, dest <chr>, air_time <dbl>, distance <dbl>,
## #   hour <dbl>, minute <dbl>, time_hour <dttm>
\end{verbatim}

\begin{enumerate}
\def\labelenumi{\alph{enumi}.}
\setcounter{enumi}{5}
\tightlist
\item
  Atrasaram pelo menos uma hora, mas compensaram mais de 30 minutos durante o trajeto.
\end{enumerate}

\begin{Shaded}
\begin{Highlighting}[]
\FunctionTok{filter}\NormalTok{(flights, dep\_delay }\SpecialCharTok{\textgreater{}=} \DecValTok{60} \SpecialCharTok{\&}\NormalTok{ dep\_delay }\SpecialCharTok{{-}}\NormalTok{ arr\_delay }\SpecialCharTok{\textgreater{}=} \DecValTok{30}\NormalTok{)}
\end{Highlighting}
\end{Shaded}

\begin{verbatim}
## # A tibble: 2,074 x 19
##     year month   day dep_time sched_dep_time dep_delay arr_time sched_arr_time
##    <int> <int> <int>    <int>          <int>     <dbl>    <int>          <int>
##  1  2013     1     1     1716           1545        91     2140           2039
##  2  2013     1     1     2205           1720       285       46           2040
##  3  2013     1     1     2326           2130       116      131             18
##  4  2013     1     3     1503           1221       162     1803           1555
##  5  2013     1     3     1821           1530       171     2131           1910
##  6  2013     1     3     1839           1700        99     2056           1950
##  7  2013     1     3     1850           1745        65     2148           2120
##  8  2013     1     3     1923           1815        68     2036           1958
##  9  2013     1     3     1941           1759       102     2246           2139
## 10  2013     1     3     1950           1845        65     2228           2227
## # i 2,064 more rows
## # i 11 more variables: arr_delay <dbl>, carrier <chr>, flight <int>,
## #   tailnum <chr>, origin <chr>, dest <chr>, air_time <dbl>, distance <dbl>,
## #   hour <dbl>, minute <dbl>, time_hour <dttm>
\end{verbatim}

\begin{enumerate}
\def\labelenumi{\alph{enumi}.}
\setcounter{enumi}{6}
\tightlist
\item
  Saíram entre meia-noite e 6h (incluindo esses horários).
\end{enumerate}

\begin{Shaded}
\begin{Highlighting}[]
\FunctionTok{filter}\NormalTok{(flights, dep\_time }\SpecialCharTok{\textgreater{}=} \DecValTok{0}\NormalTok{, dep\_time }\SpecialCharTok{\textless{}=} \DecValTok{600}\NormalTok{)}
\end{Highlighting}
\end{Shaded}

\begin{verbatim}
## # A tibble: 9,344 x 19
##     year month   day dep_time sched_dep_time dep_delay arr_time sched_arr_time
##    <int> <int> <int>    <int>          <int>     <dbl>    <int>          <int>
##  1  2013     1     1      517            515         2      830            819
##  2  2013     1     1      533            529         4      850            830
##  3  2013     1     1      542            540         2      923            850
##  4  2013     1     1      544            545        -1     1004           1022
##  5  2013     1     1      554            600        -6      812            837
##  6  2013     1     1      554            558        -4      740            728
##  7  2013     1     1      555            600        -5      913            854
##  8  2013     1     1      557            600        -3      709            723
##  9  2013     1     1      557            600        -3      838            846
## 10  2013     1     1      558            600        -2      753            745
## # i 9,334 more rows
## # i 11 more variables: arr_delay <dbl>, carrier <chr>, flight <int>,
## #   tailnum <chr>, origin <chr>, dest <chr>, air_time <dbl>, distance <dbl>,
## #   hour <dbl>, minute <dbl>, time_hour <dttm>
\end{verbatim}

\end{solution}

\hypertarget{exr3-3-2}{%
\subsection*{Exercício 3.3.2}\label{exr3-3-2}}
\addcontentsline{toc}{subsection}{Exercício 3.3.2}

Outro ajudante da filtragem do \textbf{dplyr} é \texttt{between()}. O que ele faz? Você consegue utilizá-lo para simplificar o código necessário para responder os desafios anteriores?

\begin{solution}

O \texttt{between} recebe três parâmetros e verifica se o primeiro está entre o segundo e o terceiro.

\begin{Shaded}
\begin{Highlighting}[]
\FunctionTok{filter}\NormalTok{(flights, }\FunctionTok{between}\NormalTok{(dep\_time, }\DecValTok{0}\NormalTok{, }\DecValTok{600}\NormalTok{))}
\end{Highlighting}
\end{Shaded}

\begin{verbatim}
## # A tibble: 9,344 x 19
##     year month   day dep_time sched_dep_time dep_delay arr_time sched_arr_time
##    <int> <int> <int>    <int>          <int>     <dbl>    <int>          <int>
##  1  2013     1     1      517            515         2      830            819
##  2  2013     1     1      533            529         4      850            830
##  3  2013     1     1      542            540         2      923            850
##  4  2013     1     1      544            545        -1     1004           1022
##  5  2013     1     1      554            600        -6      812            837
##  6  2013     1     1      554            558        -4      740            728
##  7  2013     1     1      555            600        -5      913            854
##  8  2013     1     1      557            600        -3      709            723
##  9  2013     1     1      557            600        -3      838            846
## 10  2013     1     1      558            600        -2      753            745
## # i 9,334 more rows
## # i 11 more variables: arr_delay <dbl>, carrier <chr>, flight <int>,
## #   tailnum <chr>, origin <chr>, dest <chr>, air_time <dbl>, distance <dbl>,
## #   hour <dbl>, minute <dbl>, time_hour <dttm>
\end{verbatim}

\end{solution}

\hypertarget{exr3-3-3}{%
\subsection*{Exercício 3.3.3}\label{exr3-3-3}}
\addcontentsline{toc}{subsection}{Exercício 3.3.3}

Quantos voos têm um \texttt{dep\_time} faltante? Que outras variáveis estão faltando? O que essas linhas podem representar?

\begin{solution}
\leavevmode

\begin{Shaded}
\begin{Highlighting}[]
\FunctionTok{count}\NormalTok{(flights, }\FunctionTok{is.na}\NormalTok{(dep\_time))}
\end{Highlighting}
\end{Shaded}

\begin{verbatim}
## # A tibble: 2 x 2
##   `is.na(dep_time)`      n
##   <lgl>              <int>
## 1 FALSE             328521
## 2 TRUE                8255
\end{verbatim}

\begin{Shaded}
\begin{Highlighting}[]
\FunctionTok{summary}\NormalTok{(}\FunctionTok{is.na}\NormalTok{(flights))}
\end{Highlighting}
\end{Shaded}

\begin{verbatim}
##     year           month            day           dep_time      
##  Mode :logical   Mode :logical   Mode :logical   Mode :logical  
##  FALSE:336776    FALSE:336776    FALSE:336776    FALSE:328521   
##                                                  TRUE :8255     
##  sched_dep_time  dep_delay        arr_time       sched_arr_time 
##  Mode :logical   Mode :logical   Mode :logical   Mode :logical  
##  FALSE:336776    FALSE:328521    FALSE:328063    FALSE:336776   
##                  TRUE :8255      TRUE :8713                     
##  arr_delay        carrier          flight         tailnum       
##  Mode :logical   Mode :logical   Mode :logical   Mode :logical  
##  FALSE:327346    FALSE:336776    FALSE:336776    FALSE:334264   
##  TRUE :9430                                      TRUE :2512     
##    origin           dest          air_time        distance      
##  Mode :logical   Mode :logical   Mode :logical   Mode :logical  
##  FALSE:336776    FALSE:336776    FALSE:327346    FALSE:336776   
##                                  TRUE :9430                     
##     hour           minute        time_hour      
##  Mode :logical   Mode :logical   Mode :logical  
##  FALSE:336776    FALSE:336776    FALSE:336776   
## 
\end{verbatim}

São 8255 voos com \texttt{dep\_time} faltante, o que pode indicar voos cancelados. As seguintes colunas também possuem dados faltantes: \texttt{dep\_delay}, \texttt{arr\_time}, \texttt{arr\_delay}, \texttt{tailnum} e \texttt{air\_time}.

\end{solution}

\hypertarget{exr3-3-4}{%
\subsection*{Exercício 3.3.4}\label{exr3-3-4}}
\addcontentsline{toc}{subsection}{Exercício 3.3.4}

Por que \texttt{NA\ \^{}\ 0} não é um valor faltante? Por que \texttt{NA\ \textbar{}\ TRUE} não é um valor faltante? Por que \texttt{FALSE\ \&\ NA} não é um valor faltante? Você consegue descobrir a regra geral? (\texttt{NA\ *\ 0} é um contraexemplo complicado!)

\begin{solution}
\texttt{NA\ \^{}\ 0} resulta em um, pois qualquer número real satisfaz essa mesma condição. A regra geral parece ser que, ao avaliar a expressão, sempre que o valor que \texttt{NA} representaria for indiferente para o resultado da expressão, então será retornado um valor diferente de \texttt{NA}.
\end{solution}

\hypertarget{ordenar-linhas-com-arrange}{%
\section{\texorpdfstring{Ordenar linhas com \texttt{arrange()}}{Ordenar linhas com arrange()}}\label{ordenar-linhas-com-arrange}}

\hypertarget{exr3-4-1}{%
\subsection*{Exercício 3.4.1}\label{exr3-4-1}}
\addcontentsline{toc}{subsection}{Exercício 3.4.1}

x

\begin{solution}
x
\end{solution}

\hypertarget{exr3-4-2}{%
\subsection*{Exercício 3.4.2}\label{exr3-4-2}}
\addcontentsline{toc}{subsection}{Exercício 3.4.2}

x

\begin{solution}
x
\end{solution}

\hypertarget{exr3-4-3}{%
\subsection*{Exercício 3.4.3}\label{exr3-4-3}}
\addcontentsline{toc}{subsection}{Exercício 3.4.3}

x

\begin{solution}
x
\end{solution}

\hypertarget{exr3-4-4}{%
\subsection*{Exercício 3.4.4}\label{exr3-4-4}}
\addcontentsline{toc}{subsection}{Exercício 3.4.4}

x

\begin{solution}
x
\end{solution}

\hypertarget{selecionar-colunas-com-select}{%
\section{\texorpdfstring{Selecionar colunas com \texttt{select()}}{Selecionar colunas com select()}}\label{selecionar-colunas-com-select}}

\hypertarget{exr3-5-1}{%
\subsection*{Exercício 3.5.1}\label{exr3-5-1}}
\addcontentsline{toc}{subsection}{Exercício 3.5.1}

x

\begin{solution}
x
\end{solution}

\hypertarget{exr3-5-2}{%
\subsection*{Exercício 3.5.2}\label{exr3-5-2}}
\addcontentsline{toc}{subsection}{Exercício 3.5.2}

x

\begin{solution}
x
\end{solution}

\hypertarget{exr3-5-3}{%
\subsection*{Exercício 3.5.3}\label{exr3-5-3}}
\addcontentsline{toc}{subsection}{Exercício 3.5.3}

x

\begin{solution}
x
\end{solution}

\hypertarget{exr3-5-4}{%
\subsection*{Exercício 3.5.4}\label{exr3-5-4}}
\addcontentsline{toc}{subsection}{Exercício 3.5.4}

x

\begin{solution}
x
\end{solution}

\hypertarget{adicionar-novas-variuxe1veis-com-mutate}{%
\section{\texorpdfstring{Adicionar novas variáveis com \texttt{mutate()}}{Adicionar novas variáveis com mutate()}}\label{adicionar-novas-variuxe1veis-com-mutate}}

\hypertarget{exr3-6-1}{%
\subsection*{Exercício 3.6.1}\label{exr3-6-1}}
\addcontentsline{toc}{subsection}{Exercício 3.6.1}

x

\begin{solution}
x
\end{solution}

\hypertarget{exr3-6-2}{%
\subsection*{Exercício 3.6.2}\label{exr3-6-2}}
\addcontentsline{toc}{subsection}{Exercício 3.6.2}

x

\begin{solution}
x
\end{solution}

\hypertarget{exr3-6-3}{%
\subsection*{Exercício 3.6.3}\label{exr3-6-3}}
\addcontentsline{toc}{subsection}{Exercício 3.6.3}

x

\begin{solution}
x
\end{solution}

\hypertarget{exr3-6-4}{%
\subsection*{Exercício 3.6.4}\label{exr3-6-4}}
\addcontentsline{toc}{subsection}{Exercício 3.6.4}

x

\begin{solution}
x
\end{solution}

\hypertarget{exr3-6-5}{%
\subsection*{Exercício 3.6.5}\label{exr3-6-5}}
\addcontentsline{toc}{subsection}{Exercício 3.6.5}

x

\begin{solution}
x
\end{solution}

\hypertarget{exr3-6-6}{%
\subsection*{Exercício 3.6.6}\label{exr3-6-6}}
\addcontentsline{toc}{subsection}{Exercício 3.6.6}

x

\begin{solution}
x
\end{solution}

\hypertarget{resumos-agrupados-com-summarize}{%
\section{\texorpdfstring{Resumos agrupados com \texttt{summarize()}}{Resumos agrupados com summarize()}}\label{resumos-agrupados-com-summarize}}

\hypertarget{exr3-7-1}{%
\subsection*{Exercício 3.7.1}\label{exr3-7-1}}
\addcontentsline{toc}{subsection}{Exercício 3.7.1}

x

\begin{solution}
x
\end{solution}

\hypertarget{exr3-7-2}{%
\subsection*{Exercício 3.7.2}\label{exr3-7-2}}
\addcontentsline{toc}{subsection}{Exercício 3.7.2}

x

\begin{solution}
x
\end{solution}

\hypertarget{exr3-7-3}{%
\subsection*{Exercício 3.7.3}\label{exr3-7-3}}
\addcontentsline{toc}{subsection}{Exercício 3.7.3}

x

\begin{solution}
x
\end{solution}

\hypertarget{exr3-7-4}{%
\subsection*{Exercício 3.7.4}\label{exr3-7-4}}
\addcontentsline{toc}{subsection}{Exercício 3.7.4}

x

\begin{solution}
x
\end{solution}

\hypertarget{exr3-7-5}{%
\subsection*{Exercício 3.7.5}\label{exr3-7-5}}
\addcontentsline{toc}{subsection}{Exercício 3.7.5}

x

\begin{solution}
x
\end{solution}

\hypertarget{exr3-7-6}{%
\subsection*{Exercício 3.7.6}\label{exr3-7-6}}
\addcontentsline{toc}{subsection}{Exercício 3.7.6}

x

\begin{solution}
x
\end{solution}

\hypertarget{exr3-7-7}{%
\subsection*{Exercício 3.7.7}\label{exr3-7-7}}
\addcontentsline{toc}{subsection}{Exercício 3.7.7}

x

\begin{solution}
x
\end{solution}

\hypertarget{mudanuxe7as-agrupadas-e-filtros}{%
\section{Mudanças agrupadas (e filtros)}\label{mudanuxe7as-agrupadas-e-filtros}}

\hypertarget{exr3-8-1}{%
\subsection*{Exercício 3.8.1}\label{exr3-8-1}}
\addcontentsline{toc}{subsection}{Exercício 3.8.1}

x

\begin{solution}
x
\end{solution}

\hypertarget{exr3-8-2}{%
\subsection*{Exercício 3.8.2}\label{exr3-8-2}}
\addcontentsline{toc}{subsection}{Exercício 3.8.2}

x

\begin{solution}
x
\end{solution}

\hypertarget{exr3-8-3}{%
\subsection*{Exercício 3.8.3}\label{exr3-8-3}}
\addcontentsline{toc}{subsection}{Exercício 3.8.3}

x

\begin{solution}
x
\end{solution}

\hypertarget{exr3-8-4}{%
\subsection*{Exercício 3.8.4}\label{exr3-8-4}}
\addcontentsline{toc}{subsection}{Exercício 3.8.4}

x

\begin{solution}
x
\end{solution}

\hypertarget{exr3-8-5}{%
\subsection*{Exercício 3.8.5}\label{exr3-8-5}}
\addcontentsline{toc}{subsection}{Exercício 3.8.5}

x

\begin{solution}
x
\end{solution}

\hypertarget{exr3-8-6}{%
\subsection*{Exercício 3.8.6}\label{exr3-8-6}}
\addcontentsline{toc}{subsection}{Exercício 3.8.6}

x

\begin{solution}
x
\end{solution}

\hypertarget{exr3-8-7}{%
\subsection*{Exercício 3.8.7}\label{exr3-8-7}}
\addcontentsline{toc}{subsection}{Exercício 3.8.7}

x

\begin{solution}
x
\end{solution}

\hypertarget{fluxo-de-trabalho-scripts}{%
\chapter{Fluxo de trabalho: scripts}\label{fluxo-de-trabalho-scripts}}

\hypertarget{anuxe1lise-exploratuxf3ria-de-dados}{%
\chapter{Análise exploratória de dados}\label{anuxe1lise-exploratuxf3ria-de-dados}}

\hypertarget{fluxo-de-trabalho-projetos}{%
\chapter{Fluxo de trabalho: projetos}\label{fluxo-de-trabalho-projetos}}

\hypertarget{part-wrangle}{%
\part{Wrangle}\label{part-wrangle}}

\hypertarget{tibbles-com-tibble}{%
\chapter{\texorpdfstring{Tibbles com \texttt{tibble}}{Tibbles com tibble}}\label{tibbles-com-tibble}}

\hypertarget{importando-dados-com-readr}{%
\chapter{\texorpdfstring{Importando dados com \texttt{readr}}{Importando dados com readr}}\label{importando-dados-com-readr}}

\hypertarget{arrumando-dados-com-tidyr}{%
\chapter{\texorpdfstring{Arrumando dados com \texttt{tidyr}}{Arrumando dados com tidyr}}\label{arrumando-dados-com-tidyr}}

\hypertarget{dados-relacionais-com-dplyr}{%
\chapter{\texorpdfstring{Dados relacionais com \texttt{dplyr}}{Dados relacionais com dplyr}}\label{dados-relacionais-com-dplyr}}

\hypertarget{strings-com-stringr}{%
\chapter{\texorpdfstring{Strings com \texttt{stringr}}{Strings com stringr}}\label{strings-com-stringr}}

\hypertarget{fatores-com-forcats}{%
\chapter{\texorpdfstring{Fatores com \texttt{forcats}}{Fatores com forcats}}\label{fatores-com-forcats}}

\hypertarget{datas-e-horas-com-lubridate}{%
\chapter{\texorpdfstring{Datas e horas com \texttt{lubridate}}{Datas e horas com lubridate}}\label{datas-e-horas-com-lubridate}}

\hypertarget{part-programar}{%
\part{Programar}\label{part-programar}}

\hypertarget{pipes-com-magrittr}{%
\chapter{\texorpdfstring{Pipes com \texttt{magrittr}}{Pipes com magrittr}}\label{pipes-com-magrittr}}

\hypertarget{funuxe7uxf5es}{%
\chapter{Funções}\label{funuxe7uxf5es}}

\hypertarget{vetores}{%
\chapter{Vetores}\label{vetores}}

\hypertarget{iterauxe7uxe3o-com-purrr}{%
\chapter{\texorpdfstring{Iteração com \texttt{purrr}}{Iteração com purrr}}\label{iterauxe7uxe3o-com-purrr}}

\hypertarget{part-modelar}{%
\chapter{(PART) Modelar}\label{part-modelar}}

\hypertarget{o-buxe1sico-de-modelos-com-modelr}{%
\chapter{\texorpdfstring{O básico de modelos com \texttt{modelr}}{O básico de modelos com modelr}}\label{o-buxe1sico-de-modelos-com-modelr}}

\hypertarget{construuxe7uxe3o-de-modelos}{%
\chapter{Construção de modelos}\label{construuxe7uxe3o-de-modelos}}

\hypertarget{muitos-modelos-com-purrr-e-broom}{%
\chapter{\texorpdfstring{Muitos modelos com \texttt{purrr} e \texttt{broom}}{Muitos modelos com purrr e broom}}\label{muitos-modelos-com-purrr-e-broom}}

\hypertarget{part-comunicar}{%
\part{Comunicar}\label{part-comunicar}}

\hypertarget{r-markdown}{%
\chapter{R Markdown}\label{r-markdown}}

\hypertarget{gruxe1ficos-para-comunicauxe7uxe3o-com-ggplot2}{%
\chapter{\texorpdfstring{Gráficos para comunicação com \texttt{ggplot2}}{Gráficos para comunicação com ggplot2}}\label{gruxe1ficos-para-comunicauxe7uxe3o-com-ggplot2}}

\hypertarget{formatos-r-markdown}{%
\chapter{Formatos R Markdown}\label{formatos-r-markdown}}

\hypertarget{fluxo-de-trabalho-de-r-markdown}{%
\chapter{Fluxo de trabalho de R Markdown}\label{fluxo-de-trabalho-de-r-markdown}}

  \bibliography{latex/book.bib,latex/packages.bib}

\printindex

\end{document}
