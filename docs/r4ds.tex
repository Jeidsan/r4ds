% Options for packages loaded elsewhere
\PassOptionsToPackage{unicode}{hyperref}
\PassOptionsToPackage{hyphens}{url}
%
\documentclass[
]{book}
\usepackage{amsmath,amssymb}
\usepackage{iftex}
\ifPDFTeX
  \usepackage[T1]{fontenc}
  \usepackage[utf8]{inputenc}
  \usepackage{textcomp} % provide euro and other symbols
\else % if luatex or xetex
  \usepackage{unicode-math} % this also loads fontspec
  \defaultfontfeatures{Scale=MatchLowercase}
  \defaultfontfeatures[\rmfamily]{Ligatures=TeX,Scale=1}
\fi
\usepackage{lmodern}
\ifPDFTeX\else
  % xetex/luatex font selection
\fi
% Use upquote if available, for straight quotes in verbatim environments
\IfFileExists{upquote.sty}{\usepackage{upquote}}{}
\IfFileExists{microtype.sty}{% use microtype if available
  \usepackage[]{microtype}
  \UseMicrotypeSet[protrusion]{basicmath} % disable protrusion for tt fonts
}{}
\makeatletter
\@ifundefined{KOMAClassName}{% if non-KOMA class
  \IfFileExists{parskip.sty}{%
    \usepackage{parskip}
  }{% else
    \setlength{\parindent}{0pt}
    \setlength{\parskip}{6pt plus 2pt minus 1pt}}
}{% if KOMA class
  \KOMAoptions{parskip=half}}
\makeatother
\usepackage{xcolor}
\usepackage{longtable,booktabs,array}
\usepackage{calc} % for calculating minipage widths
% Correct order of tables after \paragraph or \subparagraph
\usepackage{etoolbox}
\makeatletter
\patchcmd\longtable{\par}{\if@noskipsec\mbox{}\fi\par}{}{}
\makeatother
% Allow footnotes in longtable head/foot
\IfFileExists{footnotehyper.sty}{\usepackage{footnotehyper}}{\usepackage{footnote}}
\makesavenoteenv{longtable}
\usepackage{graphicx}
\makeatletter
\def\maxwidth{\ifdim\Gin@nat@width>\linewidth\linewidth\else\Gin@nat@width\fi}
\def\maxheight{\ifdim\Gin@nat@height>\textheight\textheight\else\Gin@nat@height\fi}
\makeatother
% Scale images if necessary, so that they will not overflow the page
% margins by default, and it is still possible to overwrite the defaults
% using explicit options in \includegraphics[width, height, ...]{}
\setkeys{Gin}{width=\maxwidth,height=\maxheight,keepaspectratio}
% Set default figure placement to htbp
\makeatletter
\def\fps@figure{htbp}
\makeatother
\setlength{\emergencystretch}{3em} % prevent overfull lines
\providecommand{\tightlist}{%
  \setlength{\itemsep}{0pt}\setlength{\parskip}{0pt}}
\setcounter{secnumdepth}{5}
\usepackage{booktabs}
\usepackage[portuguese]{babel}
\ifLuaTeX
  \usepackage{selnolig}  % disable illegal ligatures
\fi
\usepackage[]{natbib}
\bibliographystyle{plainnat}
\IfFileExists{bookmark.sty}{\usepackage{bookmark}}{\usepackage{hyperref}}
\IfFileExists{xurl.sty}{\usepackage{xurl}}{} % add URL line breaks if available
\urlstyle{same}
\hypersetup{
  pdftitle={R para Data Science},
  pdfauthor={Jeidsan A. da C. Pereira},
  hidelinks,
  pdfcreator={LaTeX via pandoc}}

\title{R para Data Science}
\usepackage{etoolbox}
\makeatletter
\providecommand{\subtitle}[1]{% add subtitle to \maketitle
  \apptocmd{\@title}{\par {\large #1 \par}}{}{}
}
\makeatother
\subtitle{Solução dos exercícios}
\author{Jeidsan A. da C. Pereira}
\date{2023-10-25}

\usepackage{amsthm}
\newtheorem{theorem}{Teorema}[chapter]
\newtheorem{lemma}{Lema}[chapter]
\newtheorem{corollary}{Corolário}[chapter]
\newtheorem{proposition}{Proposição}[chapter]
\newtheorem{conjecture}{Conjectura}[chapter]
\theoremstyle{definition}
\newtheorem{definition}{Definição}[chapter]
\theoremstyle{definition}
\newtheorem{example}{Exemplo}[chapter]
\theoremstyle{definition}
\newtheorem{exercise}{Exercício}[chapter]
\theoremstyle{definition}
\newtheorem{hypothesis}{Hipótese}[chapter]
\theoremstyle{remark}
\newtheorem*{remark}{Observação }
\newtheorem*{solution}{Solução }
\begin{document}
\maketitle

{
\setcounter{tocdepth}{1}
\tableofcontents
}
\begin{verbatim}
## -- Attaching core tidyverse packages ------------------------ tidyverse 2.0.0 --
## v dplyr     1.1.3     v readr     2.1.4
## v forcats   1.0.0     v stringr   1.5.0
## v ggplot2   3.4.3     v tibble    3.2.1
## v lubridate 1.9.2     v tidyr     1.3.0
## v purrr     1.0.2     
## -- Conflicts ------------------------------------------ tidyverse_conflicts() --
## x dplyr::filter() masks stats::filter()
## x dplyr::lag()    masks stats::lag()
## i Use the conflicted package (<http://conflicted.r-lib.org/>) to force all conflicts to become errors
\end{verbatim}

\hypertarget{introduuxe7uxe3o}{%
\chapter*{Introdução}\label{introduuxe7uxe3o}}
\addcontentsline{toc}{chapter}{Introdução}

Esta página serviu para estudo e prática com o pacote R Bookdown e contém a solução encontrada por mim para os exercícios propostos no livro R para Data Sciente, de Hadley Wickham e Garret Grolemund, publicado no Brasil em 2019 pela Alta Books Editora \citep{wickham2019}.

Por se tratar de um produto construído durante o processo de aprendizagem, o conteúdo pode conter erros, tanto no texto em si, como na lógica utilizada para solução dos exercícios.

Dúvidas ou sugestões de melhoria podem ser encaminhadas para o e-mail \emph{\href{mailto:jeidsan.pereira@gmail.com}{\nolinkurl{jeidsan.pereira@gmail.com}}}.

\hypertarget{part-explorar}{%
\part{Explorar}\label{part-explorar}}

\hypertarget{visualizauxe7uxe3o-de-dados-com-ggplot2}{%
\chapter{\texorpdfstring{Visualização de dados com \texttt{ggplot2}}{Visualização de dados com ggplot2}}\label{visualizauxe7uxe3o-de-dados-com-ggplot2}}

\hypertarget{introduuxe7uxe3o-1}{%
\section{Introdução}\label{introduuxe7uxe3o-1}}

Não temos exercícios nesta seção.

\hypertarget{primeiros-passos}{%
\section{Primeiros passos}\label{primeiros-passos}}

\begin{exercise}
Execute \texttt{ggplot(data=mpg);}. O que você vê?
\end{exercise}

\hypertarget{mapeamentos-estuxe9ticos}{%
\section{Mapeamentos estéticos}\label{mapeamentos-estuxe9ticos}}

\begin{exercise}
x
\end{exercise}

\hypertarget{problemas-comuns}{%
\section{Problemas comuns}\label{problemas-comuns}}

\begin{exercise}
x
\end{exercise}

\hypertarget{facetas}{%
\section{Facetas}\label{facetas}}

\begin{exercise}
x
\end{exercise}

\hypertarget{objetos-geomuxe9tricos}{%
\section{Objetos geométricos}\label{objetos-geomuxe9tricos}}

\begin{exercise}
x
\end{exercise}

\hypertarget{transformauxe7uxf5es-estatuxedsticas}{%
\section{Transformações estatísticas}\label{transformauxe7uxf5es-estatuxedsticas}}

\begin{exercise}
x
\end{exercise}

\hypertarget{ajustes-de-posiuxe7uxe3o}{%
\section{Ajustes de posição}\label{ajustes-de-posiuxe7uxe3o}}

\begin{exercise}
x
\end{exercise}

\hypertarget{sistemas-de-coordenadas}{%
\section{Sistemas de coordenadas}\label{sistemas-de-coordenadas}}

\begin{exercise}
x
\end{exercise}

\hypertarget{a-gramuxe1tica-em-camadas-de-gruxe1ficos}{%
\section{A gramática em camadas de gráficos}\label{a-gramuxe1tica-em-camadas-de-gruxe1ficos}}

\begin{exercise}
x
\end{exercise}

\hypertarget{fluxo-de-trabalho-o-buxe1sico}{%
\chapter{Fluxo de trabalho: o básico}\label{fluxo-de-trabalho-o-buxe1sico}}

\hypertarget{transformauxe7uxe3o-de-dados-com-dplyr}{%
\chapter{\texorpdfstring{Transformação de dados com \texttt{dplyr}}{Transformação de dados com dplyr}}\label{transformauxe7uxe3o-de-dados-com-dplyr}}

\hypertarget{fluxo-de-trabalho-scripts}{%
\chapter{Fluxo de trabalho: scripts}\label{fluxo-de-trabalho-scripts}}

\hypertarget{anuxe1lise-exploratuxf3ria-de-dados}{%
\chapter{Análise exploratória de dados}\label{anuxe1lise-exploratuxf3ria-de-dados}}

\hypertarget{fluxo-de-trabalho-projetos}{%
\chapter{Fluxo de trabalho: projetos}\label{fluxo-de-trabalho-projetos}}

\hypertarget{part-wrangle}{%
\part{Wrangle}\label{part-wrangle}}

\hypertarget{tibbles-com-tibble}{%
\chapter{\texorpdfstring{Tibbles com \texttt{tibble}}{Tibbles com tibble}}\label{tibbles-com-tibble}}

\hypertarget{importando-dados-com-readr}{%
\chapter{\texorpdfstring{Importando dados com \texttt{readr}}{Importando dados com readr}}\label{importando-dados-com-readr}}

\hypertarget{arrumando-dados-com-tidyr}{%
\chapter{\texorpdfstring{Arrumando dados com \texttt{tidyr}}{Arrumando dados com tidyr}}\label{arrumando-dados-com-tidyr}}

\hypertarget{dados-relacionais-com-dplyr}{%
\chapter{\texorpdfstring{Dados relacionais com \texttt{dplyr}}{Dados relacionais com dplyr}}\label{dados-relacionais-com-dplyr}}

\hypertarget{strings-com-stringr}{%
\chapter{\texorpdfstring{Strings com \texttt{stringr}}{Strings com stringr}}\label{strings-com-stringr}}

\hypertarget{fatores-com-forcats}{%
\chapter{\texorpdfstring{Fatores com \texttt{forcats}}{Fatores com forcats}}\label{fatores-com-forcats}}

\hypertarget{datas-e-horas-com-lubridate}{%
\chapter{\texorpdfstring{Datas e horas com \texttt{lubridate}}{Datas e horas com lubridate}}\label{datas-e-horas-com-lubridate}}

\hypertarget{part-programar}{%
\part{Programar}\label{part-programar}}

\hypertarget{pipes-com-magrittr}{%
\chapter{\texorpdfstring{Pipes com \texttt{magrittr}}{Pipes com magrittr}}\label{pipes-com-magrittr}}

\hypertarget{funuxe7uxf5es}{%
\chapter{Funções}\label{funuxe7uxf5es}}

\hypertarget{vetores}{%
\chapter{Vetores}\label{vetores}}

\hypertarget{iterauxe7uxe3o-com-purrr}{%
\chapter{\texorpdfstring{Iteração com \texttt{purrr}}{Iteração com purrr}}\label{iterauxe7uxe3o-com-purrr}}

\hypertarget{part-modelar}{%
\chapter{(PART) Modelar}\label{part-modelar}}

\hypertarget{o-buxe1sico-de-modelos-com-modelr}{%
\chapter{\texorpdfstring{O básico de modelos com \texttt{modelr}}{O básico de modelos com modelr}}\label{o-buxe1sico-de-modelos-com-modelr}}

\hypertarget{construuxe7uxe3o-de-modelos}{%
\chapter{Construção de modelos}\label{construuxe7uxe3o-de-modelos}}

\hypertarget{muitos-modelos-com-purrr-e-broom}{%
\chapter{\texorpdfstring{Muitos modelos com \texttt{purrr} e \texttt{broom}}{Muitos modelos com purrr e broom}}\label{muitos-modelos-com-purrr-e-broom}}

\hypertarget{part-comunicar}{%
\part{Comunicar}\label{part-comunicar}}

\hypertarget{r-markdown}{%
\chapter{R Markdown}\label{r-markdown}}

\hypertarget{gruxe1ficos-para-comunicauxe7uxe3o-com-ggplot2}{%
\chapter{\texorpdfstring{Gráficos para comunicação com \texttt{ggplot2}}{Gráficos para comunicação com ggplot2}}\label{gruxe1ficos-para-comunicauxe7uxe3o-com-ggplot2}}

\hypertarget{formatos-r-markdown}{%
\chapter{Formatos R Markdown}\label{formatos-r-markdown}}

\hypertarget{fluxo-de-trabalho-de-r-markdown}{%
\chapter{Fluxo de trabalho de R Markdown}\label{fluxo-de-trabalho-de-r-markdown}}

  \bibliography{book.bib,packages.bib}

\end{document}
